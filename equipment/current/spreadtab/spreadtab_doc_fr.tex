%  ____________________________________________________________________________
% |                                                                           |
% |                                                                           |
% |                              spreadtab v0.4b                              |
% |                                                                           |
% |                                13 mai 2012                                |
% |                                                                           |
% |___________________________________________________________________________|
%
% Ce fichier est spreadtab_doc_fr.tex, le code source de la documentation en
% français du package spreadtab.
%
% Copyright Christian Tellechea 2009-2011
% email : unbonpetit@gmail.com
%
% -------------------------------------------------------------------
% This work may be distributed and/or modified under the
% conditions of the LaTeX Project Public License, either version 1.3
% of this license or (at your option) any later version.
% The latest version of this license is in
%
%     http://www.latex-project.org/lppl.txt
%
% and version 1.3 or later is part of all distributions of LaTeX
% version 2005/12/01 or later.
% -------------------------------------------------------------------
\documentclass[a4paper,10pt]{article}
\usepackage[utf8]{inputenc}
\usepackage[T1]{fontenc}
\usepackage{libertine}
\renewcommand*\oldstylenums[1]{{\fontfamily{fxlj}\selectfont #1}}
\usepackage[scaled=0.8]{luximono}
\usepackage[a4paper,margin=2.6cm]{geometry}
\usepackage{amsmath,amssymb}
\usepackage{textcomp}
\usepackage{array}
\usepackage[table]{xcolor}
\usepackage{arydshln}
\usepackage{tabularx}
\usepackage{xcolor}
\usepackage{listings}
\usepackage{numprint}
\usepackage{xspace}
\usepackage[bottom]{footmisc}
\usepackage{spreadtab}
\usepackage[protrusion=true,final,verbose=true,babel=true]{microtype}
\usepackage{fancyhdr}
	\fancyhead[L]{}
	\fancyhead[C]{\small\bfseries\ST}
	\fancyhead[R]{\scriptsize\slshape \leftmark}
	\fancyfoot[l]{\tiny Compilé par Christian \textsc{Tellechea}, le \today.}
	\fancyfoot[c]{}
	\fancyfoot[r]{\thepage}
\usepackage[frenchb]{babel}
\frenchbsetup{StandardLists=true}
\usepackage{enumitem}
\makeatletter
\definecolor{ST@bckgcolor}{rgb}{0.87,0.9,1}
\definecolor{ST@codebckgcolor}{rgb}{0.9,0.9,0.9}
\definecolor{ST@keywordstc}{rgb}{0.7,0,0}
\definecolor{ST@keywordslatex}{rgb}{0,0,1}
\definecolor{ST@arguments}{rgb}{0,0,0}
\definecolor{ST@comments}{rgb}{0.5,0.5,0.5}
\lstset{%
	language=[AlLaTeX]TeX,float=hbp,basicstyle=\footnotesize\ttfamily,identifierstyle=\color{ST@arguments},%
	keywordstyle=\color{ST@keywordslatex},commentstyle=\itshape\color{ST@comments},%
	columns=fixed,tabsize=4,frame=single,extendedchars=false,%
	showspaces=false,showstringspaces=false,numbers=left,numberstyle=\tiny\ttfamily,%
	breaklines=true,breakindent=3em,backgroundcolor=\color{ST@bckgcolor},breakautoindent=true,%
	captionpos=t,xleftmargin=1em,xrightmargin=1em,lineskip=0pt,%
	numbersep=1em,classoffset=1,%
	morekeywords={% les macros et commandes de spreadtab
		spreadtab,endspreadtab,SThiderow,SThidecol,STsavecell,STautoround,STmessage,STsetdecimalsep,STtransposecar,STdebug,STdisplaytab,STsetdisplaymarks,%
		STnumericfieldmarker,STprintnum,STtextcell,STcopy,STtag,STmakegtag,STeol,fact,id,ifeq,ifgt,iflt,numtofrshortdate,%
		numtoengshortdate,numtofrlongdate,numtoenglongdate,numtofrmonth,%
		numtoengmonth,numtofrday,numtoengday,sum,rand,randint,sumprod,%
		frshortdatetonum,engshortdatetonum,englongdatetonum,frlongdatetonum,gcd,lcm,tag,cell,row,col},%
	keywordstyle=\color{ST@keywordstc},classoffset=0}

%%%%%%%%%%% environnement OOcalc
\newcount\cntlin
\newcount\cntcol

\newtoks\t@b
\long\def\ifremain@lines#1\\#2\@nil{%
	\csname @\ifx\@empty#2\@empty second\else first\fi oftwo\endcsname}
\long\def\subst@eol#1\\#2\@nil{\addtot@b{#1\\\cline{1-1}\hdashline}%
	\ifremain@lines#2\\\@nil{\addtot@b&\subst@eol#2\@nil}{\addtot@b{#2\end{tabular}}}}
\long\def\collect@body#1\end{\subst@eol#1\@nil\end}

\newcommand\addtot@b[1]{\t@b\expandafter{\the\t@b#1}}
\newcommand\edftot@b[1]{\edef\temp@{#1}\expandafter\addtot@b\expandafter{\temp@}}

\newenvironment{OOocalc}[2][1.5cm]{% #1=largeur colonnes #2 = nombre de colonnes
	\newcolumntype w{>{\centering\arraybackslash}m{#1}}%
	\arrayrulewidth0.4pt\dashlinedash0.4pt\relax\dashlinegap3pt
	\global\cntlin\m@ne\global\cntcol\z@
	\tabfont
	\tabcolsep0pt
	\t@b{\begin{tabular}{%
		|>{\cellcolor{tabheadcolor}%
			\global\cntcol\z@\global\advance\cntlin\@ne
			\centering\arraybackslash
			\ifnum\cntlin>\z@\tableheadformat\number\cntlin\fi}
		m{2em}|*{#2}{w:}}%
		\hline
		\rowcolor{tabheadcolor}}%
	\loop
		\ifnum\cntcol<#2
			\advance\cntcol\@ne
			\addtot@b{&\multicolumn{1}{>\centering m{#1}|}}%
			\edftot@b{{\noexpand\tableheadformat\@Alph\cntcol}}%
	\repeat
	\addtot@b{\\\hline&}%
	\collect@body}{\the\t@b}
\makeatother

% commandes de personnalisation de OOcalc
\newcommand*\tabfont{\fontfamily{phv}\selectfont\footnotesize}% police du tableau (helvetica ici)
\newcommand*\tableheadformat{\bfseries}% police des en-têtes du tableau (en gras)
\colorlet{tabheadcolor}{gray!40}% couleur de font des en-têtes du tableau
\colorlet{palepink}{pink!80}
\colorlet{lightpink}{pink!30}
\colorlet{palegreen}{green!60}
\colorlet{lightgreen}{green!30}
\newcommand\CC[1]{\cellcolor{#1}}

\newcommand\verbinline[1][]{\lstinline[breaklines=false,basicstyle=\normalsize\ttfamily,#1]}
\newcommand\ST{\textsf{spreadtab}\xspace}
\newcommand\falseverb[1]{\texttt{\detokenize{#1}}}

\usepackage[bookmarks=true,bookmarksopen=true,colorlinks=true,hyperfootnotes=false,filecolor=black,linkcolor=blue,urlcolor=magenta,pdfauthor={Christian TELLECHEA},pdftitle={spreadtab},pdfsubject={Spreadtab permet d'utiliser des fonctionnalités de tableur dans un tableau avec LaTeX},pdfkeywords={spreadtab},pdfcreator={LaTeX}]{hyperref}

\begin{document}
\parindent0pt\pagestyle{fancy}\setitemize{leftmargin=3em,topsep=0pt,parsep=0pt,itemsep=0pt,label=--}\STsetdecimalsep{,}
\begin{titlepage}
	\null\par\vfill
	\begin{center}
		\begin{minipage}{0.75\linewidth}
			\begin{center}
				\Huge\bfseries\ST\par\vspace{5pt}
				\small v\csname ST@ver\endcsname\par\vspace{35pt}
				\normalsize Manuel de l'utilisateur
			\end{center}
		\end{minipage}
	\end{center}
	\vspace{1cm}
	\begin{center}
		Christian {\sc Tellechea}\par\small
		\href{mailto:unbonpetit@gmail.com}{\texttt{\textbf{unbonpetit@gmail.com}}}\par\vspace{5pt}
		\csname ST@fr@date\endcsname
	\end{center}
	\vfill
	\begin{center}
		\begin{minipage}{0.8\linewidth}
			\noindent\hrulefill\par
			\hfill\textbf{\textit{Résumé}}\hfill{}\medskip\par\footnotesize
				Cette extension permet d'utiliser des fonctionnalités de tableur dans n'importe quel environnement \og tableau\fg{} avec \LaTeX{}.\par\smallskip
				La principale fonctionnalité étant de pouvoir écrire des formules dans les cellules d'un tableau qui font références à d'autres cellules, de calculer les formules contenues dans les cellules et d'afficher les résultats numériques de ces formules dans le tableau.\par
			\hrulefill
		\end{minipage}
	\end{center}
	\vfill{}
\end{titlepage}

\tableofcontents\newpage
\parskip\medskipamount
\section{Introduction}
\subsection{Nouveau dans la version \csname ST@ver\endcsname}
Voici les nouveautés de cette version :
\begin{itemize}
	\item avant de passer la main à l'environnement demandé par l'utilisateur, \ST ne travaille pas à l'intérieur d'un groupe de telle sorte que les calculs ne provoquent plus de ``save-stack error'' dans les tableaux de grande taille avec de nombreux calculs;
	\item les commandes \verbinline-\spreadtab- et \verbinline-\endspreadtab- peuvent délimiter un tableau;
	\item implémentation des macros fonctions ``\verbinline-tag-'', ``\verbinline-cell-'', ``\verbinline-row-'', ``\verbinline-col-'' et de la commande \verbinline-\STtag- (voir page~\pageref{tag});
	\item la commande \verbinline-\STeol{<macro>}- permet de choisir quelle séquence de contrôle marquera les fins de lignes dans les tableaux de \ST (voir page~\pageref{STeol}).
\end{itemize}

\subsection{Présentation}
Cette extension permet de construire des tableaux similaire à des feuilles de calculs. Les cellules du tableau ont des coordonnées (colonne et ligne) qui peuvent être utilisées dans des formules pour calculer des valeurs dans d'autres cellules.

Ce package nécessite le moteur $\varepsilon$-\TeX, le format \LaTeXe{} ainsi que le package \href{http://www.ctan.org/tex-archive/macros/latex/contrib/fp/}{\texttt{\textbf{fp}}} à qui sont confiés les calculs. Le package \href{http://www.ctan.org/tex-archive/macros/latex/contrib/xstring/}{\texttt{\textbf{xstring}}} est également requis dans une version \emph{supérieure ou égale} à la version \verb-v1.5d [2010/03/28]-.

J'ai souhaité dès le départ de rendre ce package compatible avec \emph{tous} les environnements de tableaux, sous réserve que les séparateurs entre colonnes soient \og\verb=&=\fg{} et les retours à la ligne soient \og\verb=\\=\fg{}. Cette contrainte forte sur la compatibilité m'a conduit à programmer \ST pour qu'il agisse d'une façon \emph{totalement indépendante} de l'environnement tableau. Ainsi, la lecture du tableau, le traitement et le calcul des formules se fait \emph{avant} que l'environnement tableau ne prenne la main et ne \og voit\fg{} le corps du tableau.

Par conséquent, \ST procède en 3 étapes :
\begin{itemize}
	\item en premier lieu, \ST lit le corps du tableau et le divise en lignes puis en cellules en reconnaissant dans chacune la présence d'une  éventuelle formule;
	\item ensuite, il procède au calcul des formules contenues dans les cellules, en ayant pris soin pour chacune de calculer auparavant les cellules dépendantes. L'ordre dans lequel les cellules doivent être calculées est déterminé par \ST. Les calculs sont faits par le package \href{http://www.ctan.org/tex-archive/macros/latex/contrib/fp/}{\texttt{\textbf{fp}}};
	\item enfin, il faut reconstruire le tableau en ayant remplacé chaque formule par la valeur numérique préalablement calculée et passer la main à l'environnement tableau spécifié par l'utilisateur.
\end{itemize}
Les deux syntaxes possibles (et équivalentes) sont les suivantes, où \verb-<nom>- représente le nom de n'importe quel environnement de type tableau disponible avec \LaTeX{} ou avec une extension :\par\nobreak
\begin{minipage}{0.48\linewidth}
\begin{lstlisting}[backgroundcolor=\color{ST@codebckgcolor}]
\begin{spreadtab}{{<nom>}{<parametres>}}
	tableau avec formules et nombres
\end{spreadtab}
\end{lstlisting}
\end{minipage}\hfill ou\hfill
\begin{minipage}{0.48\linewidth}
\begin{lstlisting}[backgroundcolor=\color{ST@codebckgcolor}]
\spreadtab{{<nom>}{<parametres>}}
	tableau avec formules et nombres
\endspreadtab
\end{lstlisting}
\end{minipage}

et après le travail de \ST, on obtient un affichage comme si l'on avait écrit :\par\nobreak
\begin{lstlisting}[backgroundcolor=\color{ST@codebckgcolor}]
\begin{<nom>}{<parametres>}
	tableau avec nombres
\end{<nom>}
\end{lstlisting}

Même si disposer de fonctionnalités ressemblant à celles d'un tableur avec \LaTeX{} est appréciable, il ne faut pas perdre de vue que les 3 étapes décrites ci-dessus prennent du temps et surtout que \falseverb{fp} est lent dans ses calculs. L'ensemble conduit donc à des temps de compilation \emph{beaucoup plus importants} qu'avec un tableau classique.

Il faut ajouter que \ST \emph{ne peut remplacer un tableur}. En effet, ses possibilités sont très limitées. De plus, surtout pour des tableaux complexes ou de grande taille, le manque d'aide visuelle devient gênant\footnote{Ceci dit, je certifie qu'avec l'habitude, cette gêne tend à s'estomper (si l'on s'en tient à des tableaux raisonnables, bien sûr).}, et la syntaxe de \ST constitue aussi un obstacle supplémentaire. L'avantage de cette extension est de pouvoir écrire \emph{dans le code \LaTeX} des tableaux comportant des calculs, alors que ces tableaux sont généralement exportés\footnote{On peut signaler les 2 principaux programmes d'exportation : \href{http://calc2latex.sourceforge.net/}{\texttt{\textbf{cacl2latex}}} pour \og calc\fg{} de Open Office, et \href{http://www.ctan.org/tex-archive/support/excel2latex/}{\texttt{\textbf{excel2latex}}} pour \og excel\fg{} de Microsoft Office.} d'une feuille de calcul d'un tableur vers le code \LaTeX. On évite ainsi les désagréments des programmes d'exportation : mise en forme souvent à retoucher pour obtenir exactement ce que l'on veut, non compatibilité avec tous les environnements de tableaux, obtention de tableaux ne contenant que les valeurs (les formules sont perdues à l'exportation), exportation à recommencer si l'on modifie un seul nombre ou formule dans le tableau.

\subsection{Motivation}
Quelques mois avant de commencer à m'attaquer à ce package, Derek \textsc{O'Connor} m'avait fait remarquer que rien n'était disponible dans le monde des extensions de \LaTeX{} pour imiter ---~ne serait-ce qu'un peu~--- le calcul de formules dans des tableaux, comme cela se fait couramment avec des tableurs. J'ai trouvé le défi intéressant et je me suis lancé dans l'écriture de ce package qui en fait, n'est qu'un exercice de programmation.

La route a été longue avant d'arriver à cette version et je tiens à remercier tout particulièrement Christophe \textsc{Casseau} pour l'intérêt qu'il a porté dès le début à ce travail et les suggestions qu'il m'a faites, ainsi que plus récemment Derek \textsc{O'Connor} pour ses conseils et pour les échanges constructifs que nous avons eus. J'adresse également mes remerciements à Andrew \textsc{Parsloe} pour la relecture et les corrections apportées à la traduction de ce manuel en anglais.

\section{Fonctionnalités courantes}
Il faut noter tout d'abord qu'à l'intérieur d'un tableau sous environnement \ST, les caractères \og\verb=:=\fg{} et \og\verb=;=\fg{} perdent leur code de catégorie actif qui leur a été attribué si vous utilisez l'option \falseverb{frenchb} du package \falseverb{babel}. Par conséquent, l'espace automatique inséré avant ces caractères sera désactivé dans un tableau.

\label{STeol}Par défaut, \ST considère qu'une ligne se termine avec ``\verb-\\-'' qui est habituel dans les tableaux. Ce marqueur de fin de ligne peut être modifié via la commande \verbinline-\STeol{<macro>}-. On peut par exemple écrire \verbinline-\STeol{\tabularnewline}-. Il faut bien comprendre que les fins de lignes qui seront insérées dans le tableau final seront toujours ``\verb-\\-'', même si le marqueur de fin de ligne lorsque \ST \emph{lit} le tableau a été modifié.

\subsection{Références absolues}
Dans le tableau, les cellules sont repérées par leur références absolues de cette façon :
\begin{itemize}
	\item la colonne est une lettre de \falseverb a à \falseverb z, \falseverb a représentant la 1\iere{} colonne de gauche : on est donc d'emblée limité à 26 colonnes, ce qui devrait suffire pour la grande majorité des cas; la lettre est insensible à la casse, elle peut donc être indifféremment minuscule ou majuscule;
	\item à la suite immédiate de la lettre, un nombre entier strictement positif représente le numéro de la ligne, la ligne numéro 1 étant la ligne du haut.
\end{itemize}
Une référence absolue s'écrit donc par exemple : \og\falseverb{b4}\fg{}, \og\falseverb{C1}\fg{} ou \og\falseverb{d13}\fg\footnote{Cette notation se retrouve dans les tableurs : la lettre représente la colonne et le nombre qui suit représente la ligne. Cet ordre est le contraire de la convention utilisée avec les matrices en mathématiques.}. Visuellement, on peut illustrer ce fonctionnement par ce genre de tableau ressemblant à un tableur, ici volontairement limité à 5 lignes et 5 colonnes :
\begin{center}
\begin{OOocalc}[6em]{5}
&&&&\\&&&&\\&&&&\\&&&&\\&&&&\\
\end{OOocalc}
\end{center}
Voici un exemple où l'on calcule la somme de chaque ligne et de chaque colonne puis, la somme totale :\par\nobreak
\begin{minipage}{0.75\linewidth}
\begin{lstlisting}
\begin{spreadtab}{{tabular}{rr|r}}
22       & 54         & a1+b1 \\
43       & 65         & a2+b2 \\
49       & 37         & a3+b3 \\
\hline
a1+a2+a3 & b1+b2+b3   & a4+b4
\end{spreadtab}
\end{lstlisting}
\end{minipage}%
\begin{minipage}{0.25\linewidth}
\centering
\begin{spreadtab}{{tabular}{rr|r}}
22       & 54         & a1+b1 \\
43       & 65         & a2+b2 \\
49       & 37         & a3+b3 \\
\hline
a1+a2+a3 & b1+b2+b3   & a4+b4
\end{spreadtab}
\end{minipage}%

Pour les matheux, voici un autre exemple où l'on calcule quelques lignes du triangle de Pascal :\par\nobreak
\begin{minipage}{0.75\linewidth}
\begin{lstlisting}
\begin{spreadtab}{{tabular}{ccccc}}
1  &       &       &       &    \\
a1 & a1    &       &       &    \\
a2 & a2+b2 & b2    &       &    \\
a3 & a3+b3 & b3+c3 & c3    &    \\
a2 & a4+b4 & b4+c4 & c4+d4 & d4
\end{spreadtab}
\end{lstlisting}
\end{minipage}%
\begin{minipage}{0.25\linewidth}
\centering
\begin{spreadtab}{{tabular}{ccccc}}
1  &       &       &       &    \\
a1 & a1    &       &       &    \\
a2 & a2+b2 & b2    &       &    \\
a3 & a3+b3 & b3+c3 & c3    &    \\
a2 & a4+b4 & b4+c4 & c4+d4 & d4
\end{spreadtab}
\end{minipage}%

\subsection{Références relatives}
Pour faire référence à une cellule, il peut être commode de spécifier sa position par rapport à la cellule où se trouve la formule. Ainsi, les coordonnées \og relatives\fg{} d'une cellule sont 2 nombres relatifs écrits selon cette syntaxe : \falseverb{[x,y]}, où \falseverb x est le décalage horizontal par rapport à la cellule contenant la formule et \falseverb y est le décalage vertical. Ainsi, \falseverb{[-2,3]} fait référence à la cellule se trouvant 2 colonnes avant (à gauche) et 3 lignes après (plus bas) la cellule où se trouve la formule.

Voici à nouveau le triangle de Pascal vu ci-dessus, mais les références sont relatives et l'environnement \og\verbinline-matrix-\fg{} du package \href{http://www.ctan.org/tex-archive/macros/latex/required/amslatex/math/}{\texttt{\textbf{amsmath}}} est utilisé :\par\nobreak
\begin{minipage}{0.82\linewidth}
\begin{lstlisting}
$
\begin{spreadtab}{{matrix}{}}
1\\
[0,-1] & [-1,-1]\\
[0,-1] & [-1,-1]+[0,-1] & [-1,-1]\\
[0,-1] & [-1,-1]+[0,-1] & [-1,-1]+[0,-1] & [-1,-1]\\
[0,-1] & [-1,-1]+[0,-1] & [-1,-1]+[0,-1] & [-1,-1]+[0,-1] & [-1,-1]
\end{spreadtab}
$
\end{lstlisting}
\end{minipage}\hfill
\begin{minipage}{0.15\linewidth}
\centering
$
\begin{spreadtab}{{matrix}{}}
1\\
[0,-1] & [-1,-1]\\
[0,-1] & [-1,-1]+[0,-1] & [-1,-1]\\
[0,-1] & [-1,-1]+[0,-1] & [-1,-1]+[0,-1] & [-1,-1]\\
[0,-1] & [-1,-1]+[0,-1] & [-1,-1]+[0,-1] & [-1,-1]+[0,-1] & [-1,-1]
\end{spreadtab}
$
\end{minipage}%

On remarque que les références relatives sont plus adaptées ici puisque seulement 2 références différentes sont utilisées : \falseverb{[0,-1]} qui se réfère à la cellule de dessus et \falseverb{[-1,-1]} qui se réfère à la cellule au Nord-Ouest de la cellule où se trouve la formule.

On peut utiliser dans une même formule un mélange de références absolues et relatives.

\subsection{Cellules de texte}
Si l'on veut mettre du texte dans une cellule, il faut indiquer à \ST que la cellule ne doit pas être calculée. Il suffit de placer quelque part dans la cellule le caractère \og\falseverb @\fg{}. Ce faisant, la cellule est ignorée par \ST et devient une cellule inerte à qui il n'est pas possible\footnote{Il y a une exception à cette règle, voir la page \protect\pageref{datetonum}.} de faire référence nulle part ailleurs dans le tableau.

Voici un exemple :\par\nobreak
\begin{lstlisting}
\begin{spreadtab}{{tabular}{|r|ccc|}}
\hline
@ valeurs de $x$ & -5       &        -1 &        4 \\
@ $f(x)=2x$      & 2*[0,-1] &  2*[0,-1] & 2*[0,-1] \\\hline
\end{spreadtab}
\end{lstlisting}
\begin{center}
\begin{spreadtab}{{tabular}{|r|ccc|}}
\hline
@ valeurs de $x$ & -5       &        -1 &        4 \\
@ $f(x)=2x$      & 2*[0,-1] &  2*[0,-1] & 2*[0,-1] \\\hline
\end{spreadtab}
\end{center}
Le caractère \og\falseverb @\fg{} est le développement de la séquence de contrôle \verbinline=\STtextcell=. On peut donc redéfinir cette séquence de contrôle en ce que l'on veut et par exemple, \verbinline-\renewcommand\STtextcell{`}- fera qu'une cellule contenant le caractère ``\verb=`='' sera comprise comme étant une cellule de texte.

De plus, si une cellule est vide ou entièrement constituée d'espaces, alors \ST la considérera comme une cellule de texte.

\subsection{Cellules mixtes}
En réalité, chaque cellule est composée de \emph{deux} champs. D'un côté le \emph{champ numérique} qui contient la formule et de l'autre le \emph{champ textuel} qui sera ignoré par \falseverb{fp} et n'entre pas en ligne de compte pour les calculs :
\begin{itemize}
	\item dans une cellule, si rien n'est précisé, la totalité de la cellule est considérée comme étant le champ numérique, et le champ textuel est vide (c'était le cas pour toutes les cellules du tableau du triangle de Pascal vu précédemment);
	\item si la cellule contient \og\falseverb{@}\fg{}, alors la totalité de la cellule est considérée comme étant le champ textuel. Le champ numérique est vide et inaccessible.
	\item si la cellule contient \og\verb-:=-\fg, alors l'argument entre accolades qui suit est le champ numérique, et tout le reste est le champ textuel. La cellule a cette structure :\par\smallskip
		\hfil\verb-<champ textuel>:={champ numérique}<suite du champ textuel>-\hfil\null\par\smallskip
	On peut changer ce marqueur en la séquence de contrôle \og\verb-\=-\fg{} par exemple, en redéfinissant la macro \verbinline-\STnumericfieldmarker- de cette façon :\par\smallskip{\centering
		\verbinline-\renewcommand\STnumericfieldmarker{\=}-\par\smallskip}
	Dans ce cas, le développement de \verb-\=- n'aurait strictement aucune importance et n'interviendrait pas dans le processus : pour \ST, il ne s'agit que d'un marqueur de début de formule qui est cherché et reconnu sans être développé.
\end{itemize}
Une fois le \og\verb-champ numérique-\fg{} calculé, lui seul et le marqueur \og\verb-:=-\fg{} seront remplacés par la valeur numérique calculée.

Il faut noter que \og\verb-:={champ numérique}-\fg{} peut se trouver à l'intérieur d'accolades et ce quelque soit le niveau d'imbrication. Par exemple, dans une cellule, on peut écrire \verb-\textbf{:={a1+1}}-. Si le champ numérique de la cellule \falseverb{a1} est 5, alors la cellule contiendra au final \verb-\textbf{6}-.

Pour fixer les idées, voici un exemple très simple :\par\nobreak
\begin{lstlisting}
\begin{spreadtab}{{tabular}{|c|c||c|}}\hline
valeur 1 : :={50} & valeur 2 : :={29} & moyenne : \textbf{:={(a1+b1)/2}}\\\hline
\end{spreadtab}
\end{lstlisting}
\begin{center}
\begin{spreadtab}{{tabular}{|c|c||c|}}\hline
valeur 1 : :={50} & valeur 2 : :={29} & moyenne : \textbf{:={(a1+b1)/2}}\\\hline
\end{spreadtab}
\end{center}
À noter également que \og\verb-:={}-\fg{}, qui définit formule vide, a le même effet que \og\verb-@-\fg dans une cellule : celle-ci est comprise comme cellule de texte. Cependant, \og\verb-@-\fg et \og\verb-:={}-\fg{} \emph{ne sont pas équivalents} car une cellule textuelle contenant ce dernier peut recevoir une formule par copie (voir section suivante), ce qui est impossible avec \og\verb-@-\fg.

\subsection{Copier des formules}
Pour éviter d'avoir à recopier des formules identiques dans des cellules voisines, \ST fournit l'instruction \verbinline-\STcopy-.

Cette commande se place dans une cellule selon cette syntaxe :
\begin{center}
\verbinline-\STcopy{>-$x$\verbinline-,v-$y$\verbinline-}{formule}-
\end{center}
où $x$ et $y$ sont des nombres positifs qui représentent des décalages horizontaux et verticaux par rapport à la cellule où se trouve l'instruction. La cellule qui contient la commande et la cellule obtenue par ces décalages définissent une plage rectangulaire de cellules qui recevront la \verb-<formule>-\footnote{La copie ne peut donc se faire que vers des cellules se trouvant à droite et plus bas que la cellule dans laquelle se trouve la commande.}. La commande \verbinline-\STcopy -\emph{ne doit pas} se trouver dans une cellule où un marqueur de champ numérique \og\verb-:=-\fg{} est présent.

Voici comment se déroule la copie : elle se fait en partant de la cellule où se trouve l'instruction, la cellule source. Pour chaque cellule cible, toutes les références dans la formule sont incrémentées pour tenir compte du décalage entre la cellule cible et la cellule source. Ainsi par exemple, mettons que la cellule source contienne la formule \verb-a1+b2+c3-. Lorsque cette formule est copiée dans une cellule cible se trouvant 2 colonnes à droite et 5 lignes en dessous, cette formule deviendra : \verb-c6+d7+e8-. La formule peut également contenir des références relatives qui, puisqu'elle sont relatives, ne seront pas modifiées.

En faisant précéder d'un \og\verb-!-\fg une coordonnée d'une référence, cette coordonnée reste inchangée lors de la copie. Reprenons l'exemple précédent avec la formule \verb-a!1+!b2+!c!3-. Si cette formule est copiée dans la cellule se trouvant 2 colonnes à droite et 5 lignes en dessous, alors, elle deviendra : \verb-c1+b7+c3-. Cette fonctionnalité est compatible avec des références relatives : mettons qu'une cellule contienne la formule \verb|[-1,!-1]+[!-1,1]+[!1,!2]| et que cette formule soit copiée vers la cellule se trouvant 2 colonnes à droite et 5 lignes en dessous. Alors, elle deviendra : \verb|[-1,-6]+[-3,1]+[-1,-3]|.

Le caractère \og\verb-!-\fg est le développement de la séquence de contrôle \verbinline-\STtransposecar-. On peut donc modifier ce caractère en tout autre avec \verbinline-\renewcommand\STtransposecar-\verb-{<caractère>}-. Le point d'exclamation, utilisé par défaut, garde son code de catégorie actif qui lui a été attribué par le package \verb-babel- chargé avec l'option \verb-frenchb-.

Dans la syntaxe \verbinline-\STcopy{>-$x$\verbinline-,v-$y$\verbinline-}{formule}-, si le nombre $x$ est absent, la copie dans le sens horizontal se fait vers la droite jusqu'au bord droit du tableau. Si $y$ est absent, la copie dans le sens vertical se fait jusqu'en bas du tableau.

Voici quelques exemples :
\begin{tabular}{cl}\hline
\verb-{>3,v1}- & copie vers les 3 colonnes à droite et 1 ligne en dessous\\
\verb-{>3}- & copie vers les 3 cellules de droite\\
\verb-{v1}- & copie vers la cellule de dessous\\
\verb-{>}- & copie vers toutes les cellules situées à droite\\
\verb-{v}- & copie vers toutes les cellules situées en dessous\\
\verb-{v,>}- & copie vers le bas et vers la droite à partir de la cellule courante\\\hline
\end{tabular}

On peut facilement générer la table de multiplication de 1 à 10 :

\begin{lstlisting}
\begin{spreadtab}{{tabular}{|c|*{10}{c}|}}
\hline
@$\times$        & 1                     & \STcopy{>}{b1+1}  & & & & & & & & \\\hline
1                & \STcopy{>,v}{!a2*b!1} &                   & & & & & & & & \\
\STcopy{v}{a2+1} &                       &                   & & & & & & & & \\
                 &                       &                   & & & & & & & & \\
                 &                       &                   & & & & & & & & \\
                 &                       &                   & & & & & & & & \\
                 &                       &                   & & & & & & & & \\
                 &                       &                   & & & & & & & & \\
                 &                       &                   & & & & & & & & \\
                 &                       &                   & & & & & & & & \\
                 &                       &                   & & & & & & & & \\\hline
\end{spreadtab}
\end{lstlisting}
\begin{center}
\begin{spreadtab}{{tabular}{|c|*{10}{c}|}}
\hline
@$\times$        & 1                     & \STcopy{>}{b1+1}  & & & & & & & & \\\hline
1                & \STcopy{>,v}{!a2*b!1} &                   & & & & & & & & \\
\STcopy{v}{a2+1} &                       &                   & & & & & & & & \\
                 &                       &                   & & & & & & & & \\
                 &                       &                   & & & & & & & & \\
                 &                       &                   & & & & & & & & \\
                 &                       &                   & & & & & & & & \\
                 &                       &                   & & & & & & & & \\
                 &                       &                   & & & & & & & & \\
                 &                       &                   & & & & & & & & \\
                 &                       &                   & & & & & & & & \\\hline
\end{spreadtab}
\end{center}

Lors d'une copie de formule, si la cellule cible contient déjà un champ numérique non vide, celui-ci n'est pas écrasé et la copie ne se fait pas.

Si des commandes \verbinline-\STcopy- se trouvant danns des plusieurs cellules départ ont une même cellule cible, la formule que cette dernière reçoit est celle correspondant à la dernière commande  \verbinline-\STcopy- rencontrée lorsque le tableau est lu de haut en bas et de droite à gauche. Dans l'exemple visuel ci-dessous, le \verbinline-\STcopy- se trouvant dans la cellule rose \falseverb{B1} a son champ cible partiellement recouvert par celui du \verbinline-\STcopy- se trouvant dans la cellule verte \falseverb{C3}. En effet, cette dernière est rencontrée plus tard lors de la lecture du tableau :
\begin{center}
\begin{OOocalc}[2.3cm]{6}
1 &\CC{palepink}\ttfamily\textbackslash STcopy \{v,>\}\{!a1+1\} &\CC{lightpink} &\CC{lightpink} &\CC{lightpink} &\CC{lightpink} \\
2 &\CC{lightpink} &\CC{lightpink} &\CC{lightpink} &\CC{lightpink} &\CC{lightpink} ~\par~ \\
3 &\CC{lightpink} &\CC{palegreen}\ttfamily\textbackslash STcopy \{>2,v1\}\{!a3*10\}&\CC{lightgreen} &\CC{lightgreen} &\CC{lightpink} \\
4 &\CC{lightpink} &\CC{lightgreen} &\CC{lightgreen} &\CC{lightgreen} &\CC{lightpink}~\par~  \\
5 &\CC{lightpink} &\CC{lightpink} &\CC{lightpink} &\CC{lightpink} &\CC{lightpink}~\par~  \\
\end{OOocalc}
\end{center}
Voici cet exemple repris ci-dessous pour mettre en évidence ces comportements. Dans cet exemple, la celule \verb-b5- qui est composée d'un champ numérique et la cellule \verb-c5- qui est mixte, leur champ numérique n'étant pas vide, il est conservé.
\begin{lstlisting}
\begin{spreadtab}{{tabular}{|*6{c|}}}\hline
1 &\STcopy{v,>}{!a1+1} &                        & & & \\\hline
2 &                    &                        & & & \\\hline
3 &                    & \STcopy{>2,v1}{!a3*10} & & & \\\hline
4 &                    &                        & & & \\\hline
5 &       -1           &       a:={0}b          & & & \\\hline
\end{spreadtab}
\end{lstlisting}
\begin{center}
\begin{spreadtab}{{tabular}{|*6{c|}}}\hline
1 &\STcopy{v,>}{!a1+1} &                        & & & \\\hline
2 &                    &                        & & & \\\hline
3 &                    & \STcopy{>2,v1}{!a3*10} & & & \\\hline
4 &                    &                        & & & \\\hline
5 &       -1           &       a:={0}b          & & & \\\hline
\end{spreadtab}
\end{center}
Comme on l'a dit à la fin du chapitre précédent, on peut également copier une formule dans une cellule de texte contenant un champ numérique vide (c'est-à-dire une cellule contenant \og\verb-:={}-\fg). Dans ce cas, la formule est copiée à l'endroit où se trouve \og\verb-:={}-\fg. Par contre, dans une cellule de texte contenant \og\verb-@-\fg, la copie ne se fait pas et la cellule reste purement textuelle.

Exemple :
\begin{lstlisting}
\begin{spreadtab}{{tabular}{|*6{c|}}}\hline
1                  & 2   & 3      & 4              & 5 & 6 \\\hline
X\STcopy{>}{a1+1}Y & @XY & X:={}Y & \textbf{:={}}  &   &   \\\hline
\end{spreadtab}
\end{lstlisting}
\begin{center}
\begin{spreadtab}{{tabular}{|*6{c|}}}\hline
1                  & 2   & 3      & 4              & 5 & 6 \\\hline
X\STcopy{>}{a1+1}Y & @XY & X:={}Y & \textbf{:={}}  &   &   \\\hline
\end{spreadtab}
\end{center}

\section{Mise en forme du tableau}
\subsection{Séparateur décimal}
Le package \verb-fp- renvoie ses résultats décimaux avec le point comme séparateur décimal. Il est possible de changer ce séparateur décimal de telle sorte que tout se passe comme si les résultats retournés par \verb-fp- en tenaient compte. L'instruction \verbinline-\STsetdecimalsep- permet de changer le séparateur décimal en n'importe quel caractère en utilisant la syntaxe :
\begin{center}
\verbinline-\STsetdecimalsep-\verb-{<caractère>}-
\end{center}
Pour une utilisation en langue française, il faut donc inclure dans le préambule cette ligne :
\begin{center}
\verbinline-\STsetdecimalsep{,}-
\end{center}
Pour les champs numériques se trouvant en mode math, il faut noter que la virgule n'est pas neutre dans ce mode. En effet, elle est considérée comme une ponctuation ce qui explique qu'elle est suivie d'une espace. Pour neutraliser cet espace, on peut mettre cette virgule entre accolades comme on le voit dans ce code :

\begin{minipage}{0.65\linewidth}
\begin{lstlisting}[escapechar=Z,backgroundcolor=\color{ST@codebckgcolor}]
3,14 n'est pas affichZéZ comme $3,14$.\par
3,14 est affichZéZ comme $3{,}14$
\end{lstlisting}
\end{minipage}\hfill
\begin{minipage}{0.35\linewidth}
3,14 n'est pas affiché comme $3,14$.\par
3,14 est affiché comme $3{,}14$
\end{minipage}%

Lorsque des cellules sont en mode math, on peut\footnote{Il est cependant préférable d'utiliser le package \texttt{numprint} pour formatter les résultats. On peut aussi modifier le code mathématique de la virgule : \texttt{\textbackslash mathcode`,="013B\textbackslash relax}. Cette modification, réservée aux utilisateurs avertis, fait entrer la virgule dans la classe 0 des signes ordinaires alors qu'elle est naturellement dans la classe 6 des signes de ponctuation.} s'inspirer de cette propriété et demander à \ST de remplacer le point décimal par une virgule entre accolades avec la commande \verbinline-\STsetdecimalsep{{,}}-. Dans ces deux tableaux où chaque cellule est en mode math, on constate que l'espacement après la virgule est neutralisé pour le second :

\begin{minipage}{0.65\linewidth}
\begin{lstlisting}
\STsetdecimalsep{,}
\begin{spreadtab}{{tabular}{|*3{>{$}r<{$}}|}}\hline
@x   & @y  & @\text{Moyenne}\\\hline
5    & -4  & (a2+b2)/2\\
-6.1 & -8  & (a3+b3)/2\\
9.85 & 3.7 & (a4+b4)/2\\\hline
\end{spreadtab}\par\smallskip
\STsetdecimalsep{{,}}
\begin{spreadtab}{{tabular}{|*3{>{$}r<{$}}|}}\hline
@x   & @y  & @\text{Moyenne}\\\hline
5    & -4  & (a2+b2)/2\\
-6.1 & -8  & (a3+b3)/2\\
9.85 & 3.7 & (a4+b4)/2\\\hline
\end{spreadtab}
\end{lstlisting}
\end{minipage}%
\begin{minipage}{0.35\linewidth}
\centering
\STsetdecimalsep{,}
\begin{spreadtab}{{tabular}{|*3{>$r<$}|}}\hline
@x   & @y  & @\text{Moyenne}\\\hline
5    & -4  & (a2+b2)/2\\
-6.1 & -8  & (a3+b3)/2\\
9.85 & 3.7 & (a4+b4)/2\\\hline
\end{spreadtab}\par\smallskip
\STsetdecimalsep{{,}}
\begin{spreadtab}{{tabular}{|*3{>$r<$}|}}\hline
@x   & @y  & @\text{Moyenne}\\\hline
5    & -4  & (a2+b2)/2\\
-6.1 & -8  & (a3+b3)/2\\
9.85 & 3.7 & (a4+b4)/2\\\hline
\end{spreadtab}
\end{minipage}

\subsection{Mise en forme des nombres}
Comme cela a été précisé, tous les calculs sont faits par le package \href{http://www.ctan.org/tex-archive/macros/latex/contrib/fp/}{\texttt{\textbf{fp}}} et sa macro \falseverb{\FPeval}\footnote{À ce propos, les notations infixe ou postfixe sont acceptées par {\ttfamily\string\FPeval} ce qui signifie que les formules dans \ST peuvent être indifféremment sous forme infixe ou postfixe. Par exemple, la formule infixe \og\falseverb{a1+b1}\fg{} est équivalente aux formules postfixes \og\falseverb{a1 b1 add}\fg{} ou \og\falseverb{a1 b1 +}\fg.}. Ce package fournit d'extraordinaires possibilités de calcul pour \TeX{} et dispose de toutes les fonctions arithmétiques, scientifiques et trigonométriques usuelles. Les calculs sont faits avec une précision de $10^{-18}$, et les 18 décimales sont affichées lorsqu'un calcul ne tombe pas juste ! Sans prendre des précautions, on peut se retrouver avec beaucoup de chiffres dans les parties décimales de certains résultats.

Pour se prémunir de ce problème, plusieurs solutions existent :
\begin{itemize}
	\item on peut utiliser le package \href{http://www.ctan.org/tex-archive/macros/latex/contrib/numprint/}{\texttt{\textbf{numprint}}} qui est ce qui se fait de mieux dans l'affichage des nombres;
	\item on peut demander à \verb=fp= d'arrondir un résultat avec sa fonction \verb-round(nombre,entier)- qui arrondit \verb=nombre= avec \verb=entier= chiffres après la virgule;
	\item on peut également demander à \ST d'arrondir \emph{tous} les nombres placés dans le tableau à une certaine précision avec la macro \verbinline-\STautoround- dont l'argument est le nombre de chiffres demandés après la virgule. Si l'argument est vide, aucun arrondi n'est fait. Si on utilise la macro étoilée \verbinline-\STautoround*-, la partie décimale est remplie si besoin avec des 0 inutiles.
\end{itemize}
Voici un exemple simple des nombres de 1 à 7 et leurs inverses, arrondis à $10^{-6}$ :\par\nobreak
\begin{lstlisting}
\STautoround{6}
\begin{spreadtab}{{tabular}{|l|*7{>{\centering\arraybackslash}m{1.35cm}|}}}
\hline
@$x$     & 1    & 2    & 3    & 4    & 5    & 6    & 7   \\\hline
@$x^{-1}$& 1/b1 & 1/c1 & 1/d1 & 1/e1 & 1/f1 & 1/g1 & 1/h1\\\hline
\end{spreadtab}\medskip

\STautoround*{6}
\begin{spreadtab}{{tabular}{|l|*7{>{\centering\arraybackslash}m{1.35cm}|}}}
\hline
@$x$     & 1    & 2    & 3    & 4    & 5    & 6    & 7   \\\hline
@$x^{-1}$& 1/b1 & 1/c1 & 1/d1 & 1/e1 & 1/f1 & 1/g1 & 1/h1\\\hline
\end{spreadtab}
\end{lstlisting}
\begin{center}
\STautoround{6}
\begin{spreadtab}{{tabular}{|l|*7{>{\centering\arraybackslash}m{1.35cm}|}}}
\hline
@$x$     & 1    & 2    & 3    & 4    & 5    & 6    & 7   \\\hline
@$x^{-1}$& 1/b1 & 1/c1 & 1/d1 & 1/e1 & 1/f1 & 1/g1 & 1/h1\\\hline
\end{spreadtab}\medskip

\STautoround*{6}
\begin{spreadtab}{{tabular}{|l|*7{>{\centering\arraybackslash}m{1.35cm}|}}}
\hline
@$x$     & 1    & 2    & 3    & 4    & 5    & 6    & 7   \\\hline
@$x^{-1}$& 1/b1 & 1/c1 & 1/d1 & 1/e1 & 1/f1 & 1/g1 & 1/h1\\\hline
\end{spreadtab}
\end{center}

Tous les nombres contenus dans le tableau final, qu'ils aient été saisis tels quels ou qu'ils proviennent du résultat d'un calcul sont traités par la macro \verbinline-\STprintnum-. Par défaut, cette macro n'a aucun effet et est programmée de la façon suivante :
\begin{center}
	\verbinline-\newcommand\STprintnum[1]{#1}-
\end{center}
Il est possible d'arrondir tous les nombres via la commande \verbinline-\numprint- du package \verb-\numprint-. Pour ce faire, il faut redéfinir la macro \verbinline-\STprintnum- de cette façon :\par\nobreak
\begin{lstlisting}
\renewcommand\STprintnum[1]{\numprint{#1}}
\nprounddigits{6}
\begin{spreadtab}{{tabular}{cccccccc}}
@$x$   & 1 &  2 & 3  & 4  & 5  & 6  &7   \\\hline
@$1/x$ &1/b1&1/c1&1/d1&1/e1&1/f1&1/g1&1/h1
\end{spreadtab}
\end{lstlisting}
\begin{center}
\renewcommand\STprintnum[1]{\numprint{#1}}
\nprounddigits{6}
\begin{spreadtab}{{tabular}{cccccccc}}
@$x$   & 1 &  2 & 3  & 4  & 5  & 6  &7   \\\hline
@$1/x$ &1/b1&1/c1&1/d1&1/e1&1/f1&1/g1&1/h1
\end{spreadtab}
\end{center}

Voici un autre exemple similaire où l'on teste si le nombre à afficher est négatif avec la commande \verbinline-\FPifneg- du package \verb-fp-. Si tel est le cas, le nombre est affiché en rouge. La commande \verbinline-\STautoround- a été préférée à \verbinline-\nprounddigits- du package \verb-numprint- puisque cette dernière ajoute des 0 inutiles. On a également remis le point décimal par défaut puisque \verbinline-\numprint- est en charge de l'affichage des nombres.
\begin{lstlisting}
\STsetdecimalsep{.}
\renewcommand\STprintnum[1]{\FPifneg{#1}\color{red}\fi\numprint{#1}}
\STautoround{6}
\begin{spreadtab}{{tabular}{cccccccc}}
@$x$   & -1 &  2 & -3 & 4  & -5 & 6  & -7 \\\hline
@$1/x$ &1/b1&1/c1&1/d1&1/e1&1/f1&1/g1&1/h1
\end{spreadtab}
\end{lstlisting}
\begin{center}
\STsetdecimalsep{.}
\renewcommand\STprintnum[1]{\FPifneg{#1}\color{red}\fi\numprint{#1}}
\STautoround{6}
\begin{spreadtab}{{tabular}{cccccccc}}
@$x$   & -1 &  2 & -3 & 4  & -5 & 6  & -7 \\\hline
@$1/x$ &1/b1&1/c1&1/d1&1/e1&1/f1&1/g1&1/h1
\end{spreadtab}
\end{center}

\subsection{Retours à la ligne et filets horizontaux}
Pour bien délimiter la fin d'une ligne, \ST est contraint de reconnaître les retours à la ligne et les filets horizontaux. Ce package permet d'utiliser dans le tableau l'argument optionnel de \verb-\\- de cette façon : \verb-\\[<dimension>]-.

Pour les filets horizontaux, on peut utiliser autant de fois que l'on veut:
\begin{itemize}
	\item \verb-\hline-;
	\item \verb=\cline{x-y}= où \falseverb x et \falseverb y sont les numéro des colonnes de départ et d'arrivée du filet;
	\item \verb=\hhline{<type>}= où \verb=<type>= est le type de ligne désirée (voir la documentation du package \href{http://www.ctan.org/tex-archive/macros/latex/required/tools/}{\texttt{\textbf{hhline}}}).
	\item n'importe quelle commande du package \href{http://www.ctan.org/tex-archive/macros/latex/contrib/booktabs/}{\texttt{\textbf{booktabs}}} : \verb-\toprule-, \verb-\midrule-, \verb-\bottomrule-, \verb-\cmidrule-, \verb-\addlinespace-, \verb-\morecmidrule- et \verb-\specialrule-. Tous les arguments de ces macros, optionnels ou pas sont gérés;
	\item \verbinline-\noalign- et son argument peuvent être placés après \verb-\\- et pris en compte.
\end{itemize}
Voici le triangle de Pascal inversé, et massacré pour l'exemple :\par\nobreak
\begin{lstlisting}
\begin{spreadtab}{{tabular}{*5c}}
[0,1] & [-1,1]+[0,1] & [-1,1]+[0,1] & [-1,1]+[0,1] & [-1,1]\\[1em]
[0,1] & [-1,1]+[0,1] & [-1,1]+[0,1] & [-1,1]       &       \\
[0,1] & [-1,1]+[0,1] & [-1,1]       &              &       \\ \hline\hline
[0,1] & [-1,1]       &              &              &       \\ \cline{2-4}
1     &              &              &              &       \\ \hline
\end{spreadtab}
\end{lstlisting}
\begin{center}
\begin{spreadtab}{{tabular}{*5c}}
[0,1] & [-1,1]+[0,1] & [-1,1]+[0,1] & [-1,1]+[0,1] & [-1,1]\\[1em]
[0,1] & [-1,1]+[0,1] & [-1,1]+[0,1] & [-1,1]       &       \\
[0,1] & [-1,1]+[0,1] & [-1,1]       &              &       \\ \hline\hline
[0,1] & [-1,1]       &              &              &       \\ \cline{2-4}
1     &              &              &              &       \\ \hline
\end{spreadtab}
\end{center}

\subsection{Masquer une ligne ou une colonne}
Parfois, une colonne ou une ligne entière est destinée à recevoir des calculs intermédiaires qui n'ont pas à être affichés dans le tableau final. Pour cela \ST dispose de deux séquences de contrôle \verbinline=\SThiderow= et \verbinline=\SThidecol= qui, lorsqu'elles sont placées dans une cellule, masquent la ligne ou la colonne dans laquelle se trouve cette cellule.

Voici un exemple :\par\nobreak
\begin{lstlisting}
\begin{spreadtab}{{tabular}{|r|ccc|}}
\hline
@ valeurs de $x$       & -1         & 0\SThidecol & 2          & 3          \\\hline
@$f(x)=2x-1$           & 2*[0,-1]-1 & 2*[0,-1]-1  & 2*[0,-1]-1 & 2*[0,-1]-1 \\
@$g(x)=x-10$\SThiderow & [0,-2]-10  & [0,-2]-10   & [0,-2]-10  & [0,-2]-10  \\
@$h(x)=1-x$            & 1-[0,-3]   & 1-[0,-3]    & 1-[0,-3]   & 1-[0,-3]   \\\hline
\end{spreadtab}
\end{lstlisting}
\begin{center}
\begin{spreadtab}{{tabular}{|r|ccc|}}
\hline
@ valeurs de $x$       & -1         & 0\SThidecol & 2          & 3          \\\hline
@$f(x)=2x-1$           & 2*[0,-1]-1 & 2*[0,-1]-1  & 2*[0,-1]-1 & 2*[0,-1]-1 \\
@$g(x)=x-10$\SThiderow & [0,-2]-10  & [0,-2]-10   & [0,-2]-10  & [0,-2]-10  \\
@$h(x)=1-x$            & 1-[0,-3]   & 1-[0,-3]    & 1-[0,-3]   & 1-[0,-3]   \\\hline
\end{spreadtab}
\end{center}
On peut observer comment on masque la ligne contenant $g(x)$ et la colonne correspondant à la valeur 0.

Il faut se souvenir que les lignes et les colonnes masquées sont \emph{invisibles} pour l'environnement tableau choisi par l'utilisateur, ce qui explique que dans le préambule du tableau, seules 4 colonnes (\falseverb{|r|ccc|}) aient été définies et non 5 comme le voit \ST.

Pour voir la différence, voici le tableau obtenu en définissant 5 colonnes et en ne masquant aucune ligne ni colonne :\par\nobreak
\begin{center}
\begin{spreadtab}{{tabular}{|r|cccc|}}
\hline
@ valeurs de $x$ & -1         & 0           & 2          & 3          \\\hline
@$f(x)=2x-1$     & 2*[0,-1]-1 & 2*[0,-1]-1  & 2*[0,-1]-1 & 2*[0,-1]-1 \\
@$g(x)=x-10$     & [0,-2]-10  & [0,-2]-10   & [0,-2]-10  & [0,-2]-10  \\
@$h(x)=1-x$      & 1-[0,-3]   & 1-[0,-3]    & 1-[0,-3]   & 1-[0,-3]   \\\hline
\end{spreadtab}
\end{center}

\subsection{Sauvegarder la valeur d'une cellule}
On peut être amené à avoir besoin de la valeur numérique d'une cellule dans le tableau pour l'afficher en dehors d'une formule ou même à l'extérieur du tableau. On doit alors utiliser la commande :\par\nobreak
\begin{center}
\verbinline-\STsavecell{<sequence de controle>}{<reference absolue>}-
\end{center}
Avec un \verbinline-\global\def-\footnote{La commande \texttt{\string\def} ne vérifie pas si la macro qu'elle définit existe déjà.}, cette commande a pour effet de sauvegarder de façon globale dans \falseverb{<sequence de controle>} le résultat de la formule contenue dans la cellule \falseverb{<reference absolue>}.

On ne peut utiliser que des références \emph{absolues}; les références relatives ne sont pas acceptées car cette commande doit se placer dans l'argument optionnel de l'environnement \verbinline-spreadtab-.

Exemple :\par\nobreak
\begin{lstlisting}
\begin{spreadtab}[\STsavecell\result{c1}]{{tabular}{|c|c|c|c|c|}}
\hline
10 & a1+10 & b1+10 & a1+b1+c1 & @cell c1 : \result\\\hline
\end{spreadtab}
\par\medskip
Voici la cellule c1 : \result
\end{lstlisting}
\begin{center}
\begin{spreadtab}[\STsavecell\result{c1}]{{tabular}{|c|c|c|c|c|}}
\hline
10 & a1+10 & b1+10 & a1+b1+c1 & @cell c1 : \result\\\hline
\end{spreadtab}
\par\medskip
Voici la cellule c1 : \result
\end{center}
Si l'on veut sauvegarder plusieurs cellules, on peut mettre autant de fois que l'on veut la commande \verbinline-\STsavecell- dans l'argument optionnel.

Exemple :\par\nobreak
\begin{lstlisting}
\begin{spreadtab}[\STsavecell\hhh{b3}\STsavecell\mmm{c3}\STsavecell\sss{d3}]{{tabular}{|rc|}}\hline
@Vitesse  (km/h) &\SThidecol&\SThidecol&\SThidecol& 35 \\
@distance (km)   &          &          &          & 180\\\hline
@Temps (h min s) & trunc(e2/e1,0) & trunc(60*(e2/e1-b3),0) & trunc(3600*(e2/e1-b3)-60*c3,1) &@\hhh\ h \mmm\ min \sss\ s\\\hline
\end{spreadtab}\par\medskip
On met au moins \hhh\ heures
\end{lstlisting}
\begin{center}
\begin{spreadtab}[\STsavecell\hhh{b3}\STsavecell\mmm{c3}\STsavecell\sss{d3}]{{tabular}{|rc|}}\hline
@Vitesse  (km/h) &\SThidecol&\SThidecol&\SThidecol& 35 \\
@distance (km)   &          &          &          & 180\\\hline
@Temps (h min s) & trunc(e2/e1,0) & trunc(60*(e2/e1-b3),0) & trunc(3600*(e2/e1-b3)-60*c3,1) &@\hhh\ h \mmm\ min \sss\ s\\\hline
\end{spreadtab}\par\medskip
On met au moins \hhh\ heures
\end{center}

\subsection{Afficher la valeur d'une cellule}
Pour afficher la valeur du champ numérique d'une cellule dans le champ textuel, on vient de voir qu'on pouvait sauvegarder cette valeur dans une séquence de contrôle et se servir de cette séquence de contrôle pour l'afficher ensuite. Le procédé est un peu fastidieux et détourne de ses intentions la commande \verbinline-\STsavecell- qui sert surtout à sauvegarder une valeur pour l'utiliser en dehors du tableau.

Pour afficher simplement la valeur du champ numérique d'une cellule dans un champ textuel, on peut utiliser la syntaxe \verb-<<référence>>- qui sera remplacé par la valeur numérique de la cellule \verb-référence- où la \verb-référence- peut être relativess ou absolue. Si ce qui est entre \verb-<<- et \verb->>- n'est pas une référence, alors rien n'est fait et \verb-<<texte>>- est laissé en l'état. La \verb-référence- ne doit contenir aucun espace. Si on écrit \verb-<< a1>>- alors, l'espace fait que la référence n'est pas reconnue et rien ne sera fait.

Exemple dans une cellule purement textuelle \verb-a3- :
\begin{lstlisting}
\begin{spreadtab}{{tabular}{lr}}
@Prix de vente                & 250  \\
@Prix d'achat                 & 216  \\\hline
@B\'en\'efice (<<b1>>-<<b2>>) & b1-b2
\end{spreadtab}
\end{lstlisting}
\begin{center}
\begin{spreadtab}{{tabular}{lr}}
@Prix de vente                & 250  \\
@Prix d'achat                 & 216  \\\hline
@B\'en\'efice (<<b1>>-<<b2>>) & b1-b2
\end{spreadtab}
\end{center}
Exemple dans la cellule mixte c1 :
\begin{lstlisting}
\begin{spreadtab}{{tabular}{|c|c||c|}}\hline
23 & 32 & Moyenne $= \frac{<<a1>>+<<b1>>}{2}= :={(a1+b1)/2}$\\\hline
\end{spreadtab}
\end{lstlisting}
\begin{center}
\begin{spreadtab}{{tabular}{|c|c||c|}}\hline
23 & 32 & Moyenne $= \frac{<<a1>>+<<b1>>}{2}= :={(a1+b1)/2}$\\\hline
\end{spreadtab}
\end{center}
Les caractères qui délimitent la référence valent "\verb-<<-" et "\verb->>-" par défaut. Ils peuvent être modifiés avec la commande \verbinline-\STsetdisplaymarks- dont les 2 arguments définissent le délimiteur de gauche et le délimiteur de droite. En écrivant \verbinline-\STsetdisplaymarks{|}{|}-, on devra écrire \verb-|référence|- pour provoquer l'affichage du champ numérique de la cellule \verb-référence-.

\subsection{Utiliser \ttfamily\textbackslash multicolumn}
Le package \ST est compatible avec la syntaxe \verbinline=\multicolumn{<nombre>}{<type>}{<contenu>}= qui fusionne \falseverb{<nombre>} cellules en une cellule de type et de contenu spécifiés dans les arguments.

En utilisant \verbinline=\multicolumn=, \ST permet même de conserver une certaine cohérence au niveau du référencement des cellules. Sur ce tableau où des cellules sont fusionnées, les cases du tableau contiennent les références absolues vues par \ST :\par\nobreak
\begin{center}
\ttfamily
\begin{tabular}{|*7{c|}}\hline
a1&b1&c1&d1&e1&f1&g1\\\hline
a2&\multicolumn{2}{l|}{b2}&d2&e2&f2&g2\\\hline
\multicolumn{3}{|l|}{a3}&d3&\multicolumn{2}{l|}{e3}&g3\\\hline
\end{tabular}
\end{center}
Ainsi, quelque soit le nombre de cellules fusionnées, la cellule suivante porte un numéro de colonne qui tient compte du nombre de cellules fusionnées.

Sur la dernière ligne, les cellules \falseverb{a3}, \falseverb{b3} et \falseverb{c3} sont fusionnées, et si la cellule \falseverb{a3 }contient une formule, les cellules \falseverb{b3} et \falseverb{c3} \emph{n'existent pas} pour \ST : on ne peut nulle part faire référence à ces cellules.

Voici un exemple où chaque nombre de la ligne du haut est le produit des 2 nombres se trouvant au dessous de lui :\par\nobreak
\begin{lstlisting}
\newcolumntype{K}[1]{@{}>{\centering\arraybackslash}p{#1cm}@{}}
\begin{spreadtab}{{tabular}{*6{K{0.5}}}}
\cline{2-5}
&\multicolumn{2}{|K{1}|}{:={a2*c2}} & \multicolumn{2}{K{1}|}{:={c2*e2}} &\\\hline
\multicolumn{2}{|K{1}}{:=8}&\multicolumn{2}{|K{1}}{:=7}&\multicolumn{2}{|K{1}|}{:=6}\\\hline
\end{spreadtab}
\end{lstlisting}
\begin{center}
\newcolumntype{K}[1]{@{}>{\centering\arraybackslash}p{#1cm}@{}}
\begin{spreadtab}{{tabular}{*6{K{0.5}}}}
\cline{2-5}
&\multicolumn{2}{|K{1}|}{:={a2*c2}} & \multicolumn{2}{K{1}|}{:={c2*e2}} &\\\hline
\multicolumn{2}{|K{1}}{:=8}&\multicolumn{2}{|K{1}}{:=7}&\multicolumn{2}{|K{1}|}{:=6}\\\hline
\end{spreadtab}
\end{center}
On remarque que le marqueur de champ numérique \og\falseverb{:=}\fg{} est nécessaire dans chaque cellule où se trouve la commande \verbinline-\multicolumn-. On comprend en effet que si ce marqueur n'existait pas, la totalite de la cellule, c'est-à-dire \verb-\multicolumn{2}{|c|}{<formule>}- serait considérée comme étant une formule.

\section{Macro-fonctions}
L'extension \falseverb{fp} fournit un nombre conséquent d'opérations et de fonctions. Malgré tout, dans le cadre d'un package comme \ST, celles-ci peuvent être insuffisantes et il est possible au programmeur averti d'écrire des macro-fonctions supplémentaires en utilisant celles fournies par \falseverb{fp}. Cette section présente les macro-fonctions pour l'instant disponibles\footnote{Bien d'autres restent à écrire ! Elles seront disponibles dans des versions futures de \ST.}. Il y aura plus de précisions sur la méthode pour programmer des macro-fonctions dans la prochaine version de ce manuel.

\subsection{Macro-fonctions mathématiques}
\subsubsection{Sommer des cellules}
La fonction \og\verbinline=sum=\fg permet de faire la somme d'une ou plusieurs plages de cellules.

Elle s'utilise ainsi : \verbinline=sum(<plage 1>;<plage 2>;...;<plage n>)=, où une plage de cellules est :
\begin{itemize}
	\item soit une cellule isolée comme \og\falseverb{a1}\fg{} ou \og\falseverb{[2,1]}\fg{};
	\item soit une zone rectangulaire délimitée par la cellule supérieure gauche et inférieure droite de cette façon : \og\verb=<cellule 1>:<cellule 2>=\fg, à condition que \og\verb=<cellule 1>=\fg se trouve \emph{avant} \og\verb=<cellule 2>=\fg{} lorsqu'on parcourt le tableau de haut en bas, de gauche à droite.\par
	Voici des exemple de plages de cellules : \og\falseverb{a2:d5}\fg{}, \og\falseverb{[-1,-1]:[2,3]}\fg{}, \og\falseverb{b4:[5,1]}\fg{}.
\end{itemize}
Dans les plages de cellules, les cellules vides ou ne contenant que du texte sont considérées comme contenant le nombre 0. Il en est de même pour les cellules masquée car fusionnées par \verbinline-\multicolumn-.

Les références relatives et absolues peuvent être utilisées conjointement, comme il semble bon à l'utilisateur. Voici un exemple où l'on fait la somme des coefficients du triangle de Pascal :\par\nobreak
\begin{minipage}{0.75\linewidth}
\begin{lstlisting}
\begin{spreadtab}{{tabular}{*5c}}
\multicolumn{5}{c}{somme=:={sum(a2:e6)}}\\
[0,1] & [-1,1]+[0,1] & [-1,1]+[0,1] & [-1,1]+[0,1] & [-1,1]  \\
[0,1] & [-1,1]+[0,1] & [-1,1]+[0,1] & [-1,1]       &         \\
[0,1] & [-1,1]+[0,1] & [-1,1]       &              &         \\
[0,1] & [-1,1]       &              &              &         \\
1     &              &              &              &
\end{spreadtab}
\end{lstlisting}
\end{minipage}%
\begin{minipage}{0.25\linewidth}
\centering
\begin{spreadtab}{{tabular}{*5c}}
\multicolumn{5}{c}{somme=:={sum(a2:e6)}}\\
[0,1] & [-1,1]+[0,1] & [-1,1]+[0,1] & [-1,1]+[0,1] & [-1,1]  \\
[0,1] & [-1,1]+[0,1] & [-1,1]+[0,1] & [-1,1]       &         \\
[0,1] & [-1,1]+[0,1] & [-1,1]       &              &         \\
[0,1] & [-1,1]       &              &              &         \\
1     &              &              &              &
\end{spreadtab}
\end{minipage}%

\subsubsection{Macro-fonction \ttfamily fact}
La macro-fonction \verbinline=fact(<nombre>)= permet de calculer la factorielle de son argument, sous réserve que celui-ci soit un entier compris entre 0 et 18 inclus pour éviter des débordements\footnote{En effet, pour \falseverb{fp} le plus grand entier est $10^{18}-1$. La factorielle de 19 dépasse ce nombre.}. Le \falseverb{<nombre>} peut aussi être une référence vers une cellule contenant un nombre entier.

Voici les factorielles de 0 à 8 :\par\nobreak
\begin{lstlisting}
\begin{spreadtab}{{tabular}{*9c}}
  0     &   1    &   2    &   3    &   4    &   5    &    6   &   7    &8\\\hline
fact(a1)&fact(b1)&fact(c1)&fact(d1)&fact(e1)&fact(f1)&fact(g1)&fact(h1)&fact(i1)
\end{spreadtab}
\end{lstlisting}
\begin{center}
\begin{spreadtab}{{tabular}{*9c}}
  0     &   1    &   2    &   3    &   4    &   5    &    6   &   7    &8\\\hline
fact(a1)&fact(b1)&fact(c1)&fact(d1)&fact(e1)&fact(f1)&fact(g1)&fact(h1)&fact(i1)
\end{spreadtab}
\end{center}

\subsubsection{Macro-fonction \ttfamily sumprod}
La fonction \verbinline=sumprod= permet de multiplier les éléments correspondants de 2 plages ou plus et ensuite, additionner ces produits.

Cette fonction s'utilise ainsi : \verbinline{sumprod(<plage 1>;<plage 2>;...;<plage n>)}. Toutes les plages rectangulaires doivent avoir les mêmes dimensions.

Voici un exemple simple où l'on calcule l'âge moyen d'un groupe d'enfants âgés de 10 à 15 ans :\par\nobreak
\begin{lstlisting}
\begin{spreadtab}{{tabular}{r*6c}}
@\^Ages & 10 & 11 & 12 & 13 & 14 & 15\\
@Nombre & 5  & 8  & 20 & 55 & 9  & 3\\\hline
@Moyenne&\multicolumn{6}{l}{:={sumprod(b1:g1;b2:g2)/sum(b2:g2)}}
\end{spreadtab}
\end{lstlisting}
\begin{center}
\begin{spreadtab}{{tabular}{r*6c}}
@\^Ages & 10 & 11 & 12 & 13 & 14 & 15\\
@Nombre & 5  & 8  & 20 & 55 & 9  & 3\\\hline
@Moyenne&\multicolumn{6}{l}{:={sumprod(b1:g1;b2:g2)/sum(b2:g2)}}
\end{spreadtab}
\end{center}
De la même façon que pour la macro-fonction \verbinline-sum-, les cellules de texte, vides ou fusionnées par un \verbinline-\multicolumn- sont considérées comme contenant le nombre 0.

\subsubsection{Nombres aléatoires}
Les macro-fonctions \verbinline-randint- et \verbinline-rand- renvoient un nombre aléatoire.

À noter : la \og graine\fg{} qui initialise la suite aléatoire dépend de la date ainsi que de la la minute à laquelle est faite la compilation. Les séries aléatoires données par cette fonction changeront entre deux compilations faites à des heures dont les minutes diffèrent. Si l'on veut vraiment avoir des nombres aléatoires indépendant du moment de compilation et donc reproductibles, il faut neutraliser la macro interne \verb-\ST@seed- et définir une graine pour \verb-fp- :\par\nobreak
\begin{lstlisting}[backgroundcolor=\color{ST@codebckgcolor}]
\makeatletter
\renewcommand\ST@seed{}% redefinit la macro interne
\makeatother
\FPseed=27% donne une graine (n'importe quel entier) a fp
\end{lstlisting}
La macro fonction  \verbinline-randint- renvoie une nombre \emph{entier} qui dépend des paramètres. La syntaxe est la suivante : \verbinline=randint([<nombre1>,]<nombre2>)=, où \verb=<nombre1>= est un entier optionnel qui vaut 0 par défaut. L'entier aléatoire renvoyé est compris dans l'intervalle \verb=[<nombre1>;<nombre2>]=.

La macro fonction \verbinline-rand()- renvoie un nombre aléatoire décimal compris entre 0 et 1 :\par\nobreak
\begin{lstlisting}
\STautoround{6}
\begin{spreadtab}{{tabular}{|l|cccc|}}\hline
@nombres dans [0;1]  &rand()        &rand()        &rand()        &rand()       \\
@nombres dans [-5;5] &randint(-5,5) &randint(-5,5) &randint(-5,5) &randint(-5,5)\\
@nombres dans [0;20] &randint(20)   &randint(20)   &randint(20)   &randint(20)  \\
\hline
\end{spreadtab}
\end{lstlisting}
\begin{center}
\STautoround{6}
\begin{spreadtab}{{tabular}{|l|cccc|}}\hline
@nombres dans [0;1]  &rand()        &rand()        &rand()        &rand()       \\
@nombres dans [-5;5] &randint(-5,5) &randint(-5,5) &randint(-5,5) &randint(-5,5)\\
@nombres dans [0;20] &randint(20)   &randint(20)   &randint(20)   &randint(20)  \\
\hline
\end{spreadtab}
\end{center}

\subsubsection{PGCD et PPCM}
Les macros fonctions "gcd" et "lcm" peremttent de calculer le Plus Grand Commun Diviseur (PGCD) et le Plus Petit Commun Multiple (PPCM) des nombres passés en arguments et séparés par des virgules :
\begin{center}
\verbinline-gcd(nombre1,nombre2,...,nombreN)-\par
\verbinline-lcm(nombre1,nombre2,...,nombreN)-
\end{center}
Exemple :
\begin{lstlisting}
\begin{spreadtab}{{tabular}{|r|r|r||c|c|}}\hline
\multicolumn3{|c||}{@Nombres}& @PGCD & @PPCM \\\hline
24 & 18 & 12 & \STcopy{v}{gcd(a2,b2,c2)} & \STcopy{v}{lcm(a2,b2,c2)}\\
15 & 10 & 25 & & \\
16 & 12 & 15 & & \\
\hline
\end{spreadtab}
\end{lstlisting}
\begin{center}
\begin{spreadtab}{{tabular}{|r|r|r||c|c|}}\hline
\multicolumn3{|c||}{@Nombres}& @PGCD & @PPCM \\\hline
24 & 18 & 12 & \STcopy{v}{gcd(a2,b2,c2)} & \STcopy{v}{lcm(a2,b2,c2)}\\
15 & 10 & 25 & & \\
16 & 12 & 15 & & \\
\hline
\end{spreadtab}
\end{center}

\subsubsection{Écriture scientifique}
La macro fonction "scitodec" permet de convertir un nombre écrit en écriture scientifique en nombre décimal compréhensible par \ST pour faire ses calculs. La syntaxe est \verbinline-scitodec(<texte>)-, où le \verb-<texte>- est :
\begin{enumerate}[label=--]
	\item une suite de caractères se présentant sous la syntaxe \verb-<mantisse>EE<exposant>-, où la \verb-<mantisse>- est un nombre décimal et l'\verb->exposant>- est un entier relatif. Les "E" peuvent être écrit en majuscule ou minuscule.
	
	Le nombre ainsi représenté est $\text{\ttfamily <mantisse>}\times10^{\text{\ttfamily <exposant>}}$
	\item une référence au champ \emph{textuel} d'une cellule qui doit contenir des caractères obéissants à la syntaxe vue au point précédent.
\end{enumerate}
Exemple :
\begin{lstlisting}
\begin{spreadtab}{{tabular}{|r|r|}}\hline
@Écritures scientifiques & @Écritures décimales \\\hline
@4EE2                    & \STcopy{v}{scitodec([-1,0])}\\
@-3.1EE-3                & \\
@15ee5                   & \\
@-0.025ee7               & \\
@2.125EE0                & \\
@3.1575EE-4              & \\\hline
\end{spreadtab}
\end{lstlisting}
\begin{center}
\begin{spreadtab}{{tabular}{|r|r|}}\hline
@Écritures scientifiques & @Écritures décimales \\\hline
@4EE2                    & \STcopy{v}{scitodec([-1,0])}\\
@-3.1EE-3                & \\
@15ee5                   & \\
@-0.025ee7               & \\
@2.125EE0                & \\
@3.1575EE-4              & \\\hline
\end{spreadtab}
\end{center}

\subsubsection{Identité}
La plus simple des macros fonctions est \og\verbinline=id(<nombre>)=\fg. Celle-ci renvoie le nombre qui se trouve entre parenthèses. Mathématiquement, elle ne sert pas à grand chose, mais avec \ST, elle permet de passer des expressions mathématiques à des macros qui n'en admettent pas. On peut ainsi s'en servir dans les plages de cellules de \verbinline=sum= par exemple.

Dans le code ci dessous, la macro \verbinline=id= est utilisée pour déterminer la plage de cellules à additionner avec \verbinline-sum-. Dans cet exemple, la valeur du champ numérique de la cellule a2 vaut 8. Et donc, "\verbinline=sum([0,-1]:[id(a2-1),-1])/a2=" est équivalent à \verbinline=sum([0,-1]:[7,-1])/8= :

\begin{lstlisting}
\begin{spreadtab}{{tabular}{r*{10}c}}
                          @Entiers & 1 & 2 & 3 & 4 & 5 & 6 & 7 & 8 & 9 & 10\\
Moyenne des :={8} premiers entiers & sum([0,-1]:[id(a2-1),-1])/a2          \\
\end{spreadtab}
\end{lstlisting}
\begin{center}
\begin{spreadtab}{{tabular}{r*{10}c}}
                          @Entiers & 1 & 2 & 3 & 4 & 5 & 6 & 7 & 8 & 9 & 10\\
Moyenne des :={8} premiers entiers & sum([0,-1]:[id(a2-1),-1])/a2          \\
\end{spreadtab}
\end{center}

\subsection{Macro-fonctions de test}
Elles sont au nombre de 3 et ont la syntaxe suivante :\par\nobreak
\begin{center}
\verbinline=ifeq(nombre1,nombre2,nombre3,nombre4)=\par
\verbinline=ifgt(nombre1,nombre2,nombre3,nombre4=\par
\verbinline=iflt(nombre1,nombre2,nombre3,nombre4)=
\end{center}
La comparaison se fait entre \falseverb{nombre1} et \falseverb{nombre2} :
\begin{itemize}
	\item test d'égalité pour \verbinline-ifeq- : \falseverb{nombre1} = \falseverb{nombre2} ?
	\item test  de supériorité stricte pour \verbinline-ifgt- : \falseverb{nombre1} > \falseverb{nombre2} ?
	\item test d'infériorité stricte pour \verbinline-iflt- : \falseverb{nombre1} < \falseverb{nombre2} ?
\end{itemize}
Si le test est vrai, \falseverb{nombre3} est retourné, sinon c'est \falseverb{nombre4}.

À titre d'exemple, voici quelques valeurs de la fonction
$f(x)=\begin{cases}
10  &\text{si }x<1\\
0   &\text{si }x=1\\
-10 &\text{si }x>1
\end{cases}
$\smallskip

\begin{lstlisting}
\begin{spreadtab}{{tabular}{|*2c|}}\hline
@$x$       & @$f(x)$               \\\hline
-0.5       & iflt([-1,0],1,10,ifeq([-1,0],1,0,-10))\\
[0,-1]+0.5 & iflt([-1,0],1,10,ifeq([-1,0],1,0,-10))\\
[0,-1]+0.5 & iflt([-1,0],1,10,ifeq([-1,0],1,0,-10))\\
[0,-1]+0.5 & iflt([-1,0],1,10,ifeq([-1,0],1,0,-10))\\
[0,-1]+0.5 & iflt([-1,0],1,10,ifeq([-1,0],1,0,-10))\\
[0,-1]+0.5 & iflt([-1,0],1,10,ifeq([-1,0],1,0,-10))\\
[0,-1]+0.5 & iflt([-1,0],1,10,ifeq([-1,0],1,0,-10))\\\hline
\end{spreadtab}
\end{lstlisting}
\begin{center}
\begin{spreadtab}{{tabular}{|*2c|}}\hline
@$x$       & @$f(x)$               \\\hline
-0.5       & iflt([-1,0],1,10,ifeq([-1,0],1,0,-10))\\
[0,-1]+0.5 & iflt([-1,0],1,10,ifeq([-1,0],1,0,-10))\\
[0,-1]+0.5 & iflt([-1,0],1,10,ifeq([-1,0],1,0,-10))\\
[0,-1]+0.5 & iflt([-1,0],1,10,ifeq([-1,0],1,0,-10))\\
[0,-1]+0.5 & iflt([-1,0],1,10,ifeq([-1,0],1,0,-10))\\
[0,-1]+0.5 & iflt([-1,0],1,10,ifeq([-1,0],1,0,-10))\\
[0,-1]+0.5 & iflt([-1,0],1,10,ifeq([-1,0],1,0,-10))\\\hline
\end{spreadtab}
\end{center}

\subsection{Macro-fonctions de date}
\subsubsection{Convertir une date en nombre avec \ttfamily frshortdatetonum}\label{datetonum}
La macro \verbinline-frshortdatetonum- permet de convertir une date de la forme 14/7/1789 en un nombre qui est en fait le nombre de jours écoulés depuis le 1\ier{} mars de l'an 0\footnote{Cet \og an 0\fg{} n'existe d'ailleurs pas, mais cela ne devrait pas être gênant pour les dates contemporaines}. Il est important de noter que cette macro-fonction requiert un argument \emph{textuel} et non pas un nombre ou le résultat d'un calcul mathématique. Par conséquent, si l'argument de cette macro-fonction fait référence à une cellule, cette cellule \emph{doit} être une cellule textuelle, c'est à dire contenant \og\verb-@-\fg{} ou \og\verb-:={}-\fg{}

Dans l'exemple ci-dessous, les deux premières lignes montrent que la cellule de gauche contient une date, et la cellule de droite fait référence à cette date pour calculer le nombre correspondant. La troisième ligne montre dans la cellule de gauche la date 0, mais surtout calcule dans la cellule de droite le nombre correspondant à la date \emph{d'aujourd'hui}, en transformant en nombre les compteurs de \TeX{} \verbinline-\day-, \verbinline-\month- et \verbinline-\year- qui contiennent les numéros des jours, mois et année courants.\par\nobreak
\begin{lstlisting}
\begin{spreadtab}{{tabular}{cc}}
@14/7/1789              & frshortdatetonum(a1)\\
1/1/2001 :={}           & frshortdatetonum(a2)\\\hline
frshortdatetonum(1/3/0) & frshortdatetonum(\number\day/\number\month/\number\year)
\end{spreadtab}
\end{lstlisting}
\begin{center}
\begin{spreadtab}{{tabular}{cc}}
@14/7/1789              & frshortdatetonum(a1)\\
1/1/2001 :={}           & frshortdatetonum(a2)\\\hline
frshortdatetonum(1/3/0) & frshortdatetonum(\number\day/\number\month/\number\year)
\end{spreadtab}
\end{center}
Une autre macro-fonction existe, elle transforme une date longue du type \og 25 décembre 1789\fg{} en un nombre. L'argument peut aussi être la séquence de contrôle \verbinline-\today- si l'on a pris le soin de charger l'extension \verb-babel- avec l'option \verb-frenchb-.\par\nobreak
\begin{lstlisting}[escapechar=Z]
\begin{spreadtab}{{tabular}{cc}}
frlongdatetonum(\today) & frlongdatetonum(25 dZéZcembre 2009)\\
@1 juillet 1970         & frlongdatetonum(a2)
\end{spreadtab}
\end{lstlisting}
\begin{center}
\begin{spreadtab}{{tabular}{cc}}
frlongdatetonum(\today) & frlongdatetonum(25 décembre 2009)\\
@1 juillet 1970         & frlongdatetonum(a2)
\end{spreadtab}
\end{center}

\subsubsection{Passer d'un nombre à une date}
Plusieurs macro-fonctions permettent de traduire un nombre en une donnée de date. Toutes ces macro-fonctions ont en commun que leur résultat est du \emph{texte}. Par conséquent, \emph{la cellule les contenant deviendra une cellule textuelle} dont le texte sera le résultat de cette fonction. Si un texte cohabitait avec la formule dans cette cellule, le résultat de la formule sera inséré à la place de \verb-:={<formule>}-. Il en résulte que la cellule ne peut plus faire l'objet d'aucun traitement mathématique ensuite.

Voici ces fonctions :
\begin{itemize}
	\item \verbinline-numtofrshortdate- transforme un nombre en une date courte du type 14/7/1789;
	\item \verbinline-numtofrlongdate- transforme un nombre en une date longue du type \og 14 juillet 1789\fg;
	\item \verbinline-numtofrmonth- extrait d'un nombre représentant une date le nom du mois correspondant;
	\item \verbinline-numtofrday- extrait d'un nombre représentant une date le nom du jour correspondant.
\end{itemize}
Voici un  exemple où l'on se place 1000 jours avant puis 1000 jours après le 1/6/2009. Pour chacune de ces 2 dates, on calcule la date courte, la date longue, le mois et le jour de la semaine.\par\nobreak
\begin{lstlisting}
\begin{spreadtab}{{tabular}{cc}}                         \hline
\multicolumn{2}{|c|}{@1/6/2009}                          \\\hline\hline
1000      & numtofrshortdate(frshortdatetonum(a1)+[-1,0])\\
1000      & numtofrlongdate(frshortdatetonum(a1)+[-1,0]) \\
1000      & numtofrmonth(frshortdatetonum(a1)+[-1,0])    \\
1000      & numtofrday(frshortdatetonum(a1)+[-1,0])      \\\hline
-1000     & numtofrshortdate(frshortdatetonum(a1)+[-1,0])\\
-1000     & numtofrlongdate(frshortdatetonum(a1)+[-1,0]) \\
-1000     & numtofrmonth(frshortdatetonum(a1)+[-1,0])    \\
-1000     & numtofrday(frshortdatetonum(a1)+[-1,0])
\end{spreadtab}
\end{lstlisting}
\begin{center}
\begin{spreadtab}{{tabular}{cc}}                         \hline
\multicolumn{2}{|c|}{@1/6/2009}                          \\\hline\hline
1000      & numtofrshortdate(frshortdatetonum(a1)+[-1,0])\\
1000      & numtofrlongdate(frshortdatetonum(a1)+[-1,0]) \\
1000      & numtofrmonth(frshortdatetonum(a1)+[-1,0])    \\
1000      & numtofrday(frshortdatetonum(a1)+[-1,0])      \\\hline
-1000     & numtofrshortdate(frshortdatetonum(a1)+[-1,0])\\
-1000     & numtofrlongdate(frshortdatetonum(a1)+[-1,0]) \\
-1000     & numtofrmonth(frshortdatetonum(a1)+[-1,0])    \\
-1000     & numtofrday(frshortdatetonum(a1)+[-1,0])
\end{spreadtab}
\end{center}

\subsection{Macro fonctions de coordonnées}\label{tag}
Plutôt que de faire référence à une cellule par ses coordonnées qui sont difficiles à mémoriser et qui changent si on insère une ligne ou une colonne, il est parfois bien plus pratique de donner un nom à une cellule et d'y faire référence plus loin par ce nom.

La macro fonction ``\verbinline-tag(<nom>)-'' a pour effet de nommer la cellule dans laquelle elle se trouve. Il ne s'agit pas vraiment d'une macro fonction comme les autres puisque ce qu'elle retourne rien lorsqu'elle est mise dans une formule : elle disparait sans provoquer aucun effet sur le résultat mathématique. On peut donc écrire ``\verbinline-tag(<nom>)-'' \emph{n'importe où} dans le champ numérique d'une cellule. Le  \verb-<nom>- peut être n'importe quelle suite de caractères alphanumériques, mais il est déconseillé de mettre une lettre et un nombre, qui pourraient être compris comme une référence à une cellule, et serait donc modifié lors d'une opération de copie avec \verbinline-\STcopy-. Cette macro fonction a une action supplémentaire, elle sauvegarde via un \verbinline-\def- la valeur numérique de la cellule dans laquelle elle se trouve de façon à pouvoir y faire appel plus tard \emph{en dehors du tableau} via la commande purement développable \verbinline-\STtag{<nom>}-.

Par la suite dans le tableau, au lieu de mettre les coordonnées de la cellule \verb-<nom>-, on peut écrire ``\verbinline-cell(<nom>)-'' qui est une macro fonction qui renvoit les coordonnées de la cellule précédemment nommée. Par exemple si ``\verbinline-tag(<nom>)-'' a été écrit dans la cellule ``\verb-B3-'' si on a écrit plus loin ``cell(<nom>)'', cette macro fonction renvoit \verb-B3-.

Voici un exemple où l'on fait la somme de cellules et l'on donne le nom ``\verb-foo-'' au premier nombre et le nom ``\verb-bar-'' au dernier. On peut observer que \verbinline-tag(foo)- se trouve entre ``1'' et ``9'' et qu'au final, puisque cette macro fonction disparait, le champ numérique de la cellule contiendra ``19'' :
\begin{lstlisting}
\begin{spreadtab}{{tabular}{r@{}r}}
   & 15tag(foo)              \\
@+ & 37                      \\
@+ & 13                      \\
@+ & 48                      \\
@+ & 1tag(bar)9              \\\cline{2-2}
   & sum(cell(foo):cell(bar))tag(baz)
\end{spreadtab}

foo=\STtag{foo}, bar=\STtag{bar}, baz=\STtag{baz}
\end{lstlisting}
\begin{center}
\begin{spreadtab}{{tabular}{r@{}r}}
   & 15tag(foo)              \\
@+ & 37                      \\
@+ & 13                      \\
@+ & 48                      \\
@+ & 1tag(bar)9              \\\cline{2-2}
   & sum(cell(foo):cell(bar))tag(baz)
\end{spreadtab}

foo=\STtag{foo}, bar=\STtag{bar}, baz=\STtag{baz}
\end{center}
Lorsque l'environnement \ST se trouve inclus dans un autre environnement, les affectations faites par la macro fonction \verb-tag- restent \emph{locales} à cet environnement et ne pourraient pas être accessibles en dehors de cet environnement via \verbinline-\STtag-. Dans l'exemple ci-dessous, le tableau fait par \ST est dans un environnement \verb-center- et l'on doit donc utiliser \verbinline-\STmakegtag{<nom>}- pour rendre global la sauvegarde de la valeur numérique contenus dans la cellule marquée par ``tag{<nom>}'' :

\begin{minipage}{.6\linewidth}
\begin{lstlisting}
\begin{center}
  \begin{spreadtab}{{tabular}{cccc}}\hline
    6&9&17&21\\\hline
    \multicolumn{4}{c}{moyenne = :={sum(a1:d1)/4tag(moy)}}
  \end{spreadtab}
  \STmakegtag{moy}
\end{center}
La moyenne est de \STtag{moy}.
\end{lstlisting}
\end{minipage}\kern6pt
\begin{minipage}{.3\linewidth}
\begin{center}
  \begin{spreadtab}{{tabular}{cccc}}\hline
    6&9&17&21\\\hline
    \multicolumn{4}{c}{moyenne = :={sum(a1:d1)/4tag(moy)}}
  \end{spreadtab}
  \STmakegtag{moy}
\end{center}
La moyenne est de \STtag{moy}.
\end{minipage}

L'argument de \verbinline-\STmakegtag- peut êre constitué de plusieurs noms séparés par des virgules.

Bien qu'elles soient à première vue moins utiles, \ST fournit également les macros fonctions ``\verbinline-row(<nom>)-'' et ``\verbinline-col(<nom>)-'' qui renvoient le numéro de la ligne ou de la colonne où se trouvait \verbinline-tag(<nom>)-. Voici par exemple une façon de dénombrer des valeurs pour en calculer la moyenne : on nomme ``un'' la première valeur et ``der'' la dernière, et le nombre de valeurs est donc row(der)-row(un)+1 :

\begin{minipage}{0.6\linewidth}
\begin{lstlisting}
moyenne = \begin{spreadtab}{{tabular}[b]{r}}
7tag(un)\\9\\15\\6\\20\\13\\11\\55tag(der)\\
\hline
sum(cell(un):cell(der))/(row(der)-row(un)+1)
\end{spreadtab}
\end{lstlisting}
\end{minipage}\hfill
\begin{minipage}{0.35\linewidth}
moyenne = \begin{spreadtab}{{tabular}[b]{r}}
7tag(un)\\9\\15\\6\\20\\13\\11\\55tag(der)\\
\hline
sum(cell(un):cell(der))/(row(der)-row(un)+1)
\end{spreadtab}
\end{minipage}

\section{Précautions particulières}
\subsection{Redéfinition des commandes de filets horizontaux}
On peut être tenté de définir une commande pour produire ---~par exemple~--- une double ligne horizontale :\par\nobreak
\begin{center}
\verbinline-\newcommand\dline{\hline\hline}-
\end{center}
puis essayer de l'utiliser dans un tableau pour produire ce simple exemple qui calcule à la seconde ligne les termes de la suite de Fibonacci :\par\nobreak
\begin{center}
\newcommand\dline{\hline\hline}
\begin{spreadtab}{{tabular}{*7c}}
0     & 1 & 2     & 3     & 4     & 5     & 6     \\\dline
:={1} & 1 & a2+b2 & b2+c2 & c2+d2 & d2+e2 & e2+f2
\end{spreadtab}
\end{center}
Mais en écrivant ce code, il y a un problème :\par\nobreak
\begin{lstlisting}
\newcommand\dline{\hline\hline}
\begin{spreadtab}{{tabular}{*7c}}
0 & 1 & 2     & 3     & 4     & 5     & 6     \\\dline
1 & 1 & a2+b2 & b2+c2 & c2+d2 & d2+e2 & e2+f2
\end{spreadtab}
\end{lstlisting}
En effet la compilation échoue et dans le log, on peut lire que \verb-\FPeval- se plaint :\par
\hfill\falseverb{! Improper alphabetic constant.}\hfill\null

La raison est simple, c'est que le \verb-\dline- de la ligne 4, n'est \emph{pas} reconnu par \ST comme un filet horizontal et \emph{il se retrouve donc dans la cellule de la ligne suivante}. Pour \ST, cette cellule \falseverb{b1} contient :\par
\hfill\falseverb{\dline 1}\hfill\null

Comme il n'y a pas de \verb-@- ni de délimiteur de formule \verb-:={...}-, \verb-\FPeval- essaie vaillamment de calculer ce contenu et échoue évidemment !

Pour pouvoir compiler ce code sans erreur, la cellule \falseverb{a2} \emph{doit} contenir un marqueur de champ numérique :\par\nobreak
\begin{lstlisting}
\newcommand\dline{\hline\hline}
\begin{spreadtab}{{tabular}{*7c}}
0     & 1 & 2     & 3     & 4     & 5     & 6     \\\dline
:={1} & 1 & a2+b2 & b2+c2 & c2+d2 & d2+e2 & e2+f2
\end{spreadtab}
\end{lstlisting}
\begin{center}
\newcommand\dline{\hline\hline}
\begin{spreadtab}{{tabular}{*7c}}
0     & 1 & 2     & 3     & 4     & 5     & 6     \\\dline
:={1} & 1 & a2+b2 & b2+c2 & c2+d2 & d2+e2 & e2+f2
\end{spreadtab}
\end{center}

\subsection{Cohabitation de {\ttfamily\textbackslash multicolumn} et  \ttfamily\textbackslash SThidecol}
Tout d'abord, dans une utilisation normale, l'utilisation conjointe de \verbinline|\multicolumn| et \verbinline-\SThiderow- ne doit pas arriver, et la plupart des utilisateurs ne devrait pas rencontrer cette situation ni lire ce chapitre.

Pour les courageux venons-en au c\oe ur du problème : tout d'abord, une colonne masquée ne doit \emph{jamais} contenir une cellule où se trouve la commande \verbinline-\multicolumn- ! Mais que se passe t-il si une colonne masquée cache des cellules fusionnées par \verbinline-\multicolumn-?

Déjà, en général, il n'y a pas d'erreur de compilation ni message d'erreur, mais il y a quelques subtilités quant aux références qui sont un peu chamboulées dans la ligne concernée après le \verbinline-\multicolumn-\ldots

Prenons un exemple, et mettons que dans le tableau suivant, on fusionne les cellules \falseverb{b2} à \falseverb{h2} et que l'on souhaite cacher les colonnes \falseverb{c}, \falseverb{d} et \falseverb{f}, ici en gris :\par\nobreak
\begin{center}
\ttfamily
\begin{tabular}{|c|c|c|c|c|c|c|c|c|c|}
\hline
 a1 & b1 & \cellcolor[gray]{0.6}c1 & \cellcolor[gray]{0.6}d1 & e1 & \cellcolor[gray]{0.6}f1 & g1 & h1 & i1 & j1\\\hline
 a2 & \multicolumn1c{b2} & \multicolumn1{>{\cellcolor[gray]{0.6}}c}{} & \multicolumn1{>{\cellcolor[gray]{0.6}}c}{} & \multicolumn1c{} & \multicolumn1{>{\cellcolor[gray]{0.6}}c}{} & \multicolumn1{c}{} & \multicolumn1{c|}{} & i2 & j2\\\hline
\end{tabular}
\end{center}
Il y a 4 cellules \emph{visibles} fusionnées, on écrira donc \verbinline-\multicolumn{4}- car on ne tient \emph{jamais} compte des colonnes masquées dans le décompte du nombre de cellules à fusionner.

Maintenant, on compte 4 lettres à partir de la lettre \falseverb{b} en l'incluant dans le décompte. On arrive à la lettre \falseverb{e} : cela détermine un intervalle de colonnes \og\falseverb{b-e}\fg. Dans cet intervalle, 2 colonnes masquées sont incluses (\falseverb c et \falseverb d) et 1 colonne masquée n'est pas comprise (\falseverb f). Ces 2 nombres sont importants pour comprendre la suite, aussi, notons-les $x$ et $y$ dans le cas général.

La règle est la suivante :
\begin{itemize}
	\item il faut rajouter $y$ signes \og\verb-&-\fg{} après le \verbinline-\multicolumn- (pour l'exemple, il en faudrait 1).
	\item les références des colonnes des cellules qui suivent le \verbinline-\multicolumn- seront décalées de $x$ lettres vers le début de l'alphabet. Pour l'exemple donnée, si on veut faire référence à la cellule marquée \og \falseverb{i2}\fg{}, il faudra écrire \falseverb{g2} (au lieu de \falseverb{i2}).
\end{itemize}
Voici un vrai exemple dont la structure est similaire au précédent : $x=2$ et $y=1$. Dans le code, remarquer le \og\textcolor{red}{\texttt{\&}}\fg{} qui a été ajouté puisque $y=1$. Par ailleurs, on reste simple avec les formules, on ajoute 1 au nombre du dessus :\par\nobreak
\begin{lstlisting}[escapechar=Z]
\begin{spreadtab}{{tabular}{|*{7}{c|}}}
\hline
1   & 2    & \SThidecol3 & \SThidecol4 & 5& \SThidecol6 & 7& 8& 9    & 10  \\\hline
a1+1& \multicolumn4{l|}{:={b1+1}}Z\ttfamily\color{red}\rlap\&Z                             & i1+1 & j1+1\\\hline
a2+1& b2+1 &             &             &  &             &  &  & g2+1 & h2+1\\\hline
\end{spreadtab}
\end{lstlisting}
\begin{center}
\begin{spreadtab}{{tabular}{|*{7}{c|}}}
\hline
1   & 2    & \SThidecol3 & \SThidecol4 & 5& \SThidecol6 & 7& 8& 9    & 10  \\\hline
a1+1& \multicolumn4{l|}{:={b1+1}}&                            & i1+1 & j1+1\\\hline
a2+1& b2+1 &             &             &  &             &  &  & g2+1 & h2+1\\\hline
\end{spreadtab}
\end{center}
Voici encore un exemple similaire où une seule colonne est masquée (la colonne \falseverb d), et où $x=1$ et $y=0$ :\par\nobreak
\begin{lstlisting}
\begin{spreadtab}{{tabular}{|*{9}{c|}}}
\hline
1   & 2    & 3 & \SThidecol4 & 5 & 6 & 7 & 8    & 9    & 10  \\\hline
a1+1& \multicolumn6{l|}{:={b1+1}}               & i1+1 & j1+1\\\hline
a2+1& b2+1 &   &             &   &   &   &      & h2+1 & i2+1\\\hline
\end{spreadtab}
\end{lstlisting}
\begin{center}
\begin{spreadtab}{{tabular}{|*{9}{c|}}}
\hline
1   & 2    & 3 & \SThidecol4 & 5 & 6 & 7 & 8    & 9    & 10  \\\hline
a1+1& \multicolumn6{l|}{:={b1+1}}               & i1+1 & j1+1\\\hline
a2+1& b2+1 &   &             &   &   &   &      & h2+1 & i2+1\\\hline
\end{spreadtab}
\end{center}

\subsection{Messages émis par \ST}
Le package émet des messages d'erreur et arrête la compilation dans ces cas :
\begin{itemize}
	\item Pour calculer une cellule, on calcule de proche en proche des cellules qui par le jeu des références, reviennent sur la cellule d'origine (références circulaires). Dans ce cas, l'arbre des dépendances est affiché dans le message d'erreur;
	\item une formule nécessitant un nombre fait référence à une cellule vide ou ne contenant que du texte;
	\item une cellule fait référence à une cellule non définie (hors limite du tableau);
	\item une cellule fait référence à une cellule fusionnée par un \verbinline-\multicolumn-;
	\item une référence relative entre crochets ne respecte pas la syntaxe.
\end{itemize}

Le package peut émettre des messages d'information (dans le fichier de log), ce qu'il fait par défaut. La commande \verbinline-\STmessage- permet ou pas l'émission de messages d'information. Pour chacune des ces altenatives, la syntaxe est la suivante : \verbinline-\STmessage{true}- ou \verbinline-\STmessage{false}-.

Pour comprendre la signification des messages, prenons un tableau simple :\par\nobreak
\begin{minipage}{0.65\linewidth}
\begin{lstlisting}
\begin{spreadtab}{{tabular}{|cccc|c|}}\hline
b1+1 & c1+1 & d1+1 & 10 & a1+b1+c1+d1\\\hline
\end{spreadtab}
\end{lstlisting}
\end{minipage}%
\begin{minipage}{0.35\linewidth}
\centering
\begin{spreadtab}{{tabular}{|cccc|c|}}\hline
b1+1 & c1+1 & d1+1 & 10 & a1+b1+c1+d1\\\hline
\end{spreadtab}
\end{minipage}

Le fonctionnement du tableau est ici très simple à comprendre. Voici les informations délivrées par \ST :\par\nobreak
\begin{lstlisting}
[spreadtab] New spreadtab {tabular}{|cccc|c|}
* reading tab: ok
* computing formulas:
     cell A1-B1-C1
     cell B1
     cell C1
     cell D1
     cell E1
* building tab: ok
[spreadtab] End of spreadtab
\end{lstlisting}
L'environnement spécifié par l'utilisateur est repris entre parenthèses (ici \verbinline-{tabular}{|cccc|c|}-). Précédées d'une étoile, on retrouve les 3 étapes nécessaires à \ST pour mener à bien sa mission : lecture du tableau, calcul des formules et construction du tableau final.

Pour la seconde étape, les cellules sont évaluées de haut en bas, de gauche à droite : \ST indique qu'il commence par essayer de calculer la première cellule \falseverb{A1}. Pour cela, il indique qu'il doit d'abord évaluer \falseverb{B1} et avant cela encore, évaluer \falseverb{C1}. Comme il n'y a plus de cellule après \falseverb{C1}, c'est qu'elle peut être évaluée; en effet, elle ne dépend que de \falseverb{D1} qui est un nombre égal à 10.

Pour chaque ligne suivante, il n'y a qu'une seule cellule ce qui signifie que lorsque \ST essaie de les évaluer, elles l'ont déjà été et sont des nombres ou alors, elles ne font références qu'à des cellules déjà calculées.

\subsection{Débogage}
Pour faciliter l'utilisation de \ST, un mode de débogage est disponible. Il est activé lorsque la commande \verbinline-\STdebug- est présente dans l'argument optionnel de l'environnement \verb-spreadtab-. Cette commande change le comportement de \ST qui, au lieu d'afficher le tableau final, affiche un (ou plusieurs) tableau de débogage. Cet affichage se fait juste après que \ST ait lu l'ensemble de toutes les cellules ; aucun calcul de formule n'a encore eu lieu. Il y a autant de tableaux affichés que de commande \verbinline-\STdebug- présentes, sous réserve que leur argument soit différent. Seuls 3 arguments sont possibles, et voici ce qu'il permettent :
\begin{itemize}
	\item \verbinline-\STdebug{formula}- : affichage des champs numériques de toutes les cellules et les fins de ligne;
	\item \verbinline-\STdebug{text}- : affichage des champs textuels de toutes les cellules;
	\item \verbinline-\STdebug{code}- : affichage du code interne que \ST affecte à chaque cellule. Ce code interne est assigné à chaque cellule lors de la lecture du tableau. Il vaut :
	\begin{enumerate}[label=\small\textbullet]
		\item $-1$ s'il s'agit d'une cellule fusionnée par une commande \verbinline-\multicolumn-;
		\item 0 si la cellule est vide ou purement textuelle;
		\item 1 si le champ numérique de la cellule contient une formule qui sera à évaluer plus tard;
		\item 2 si la champ numérique de la cellule contient un nombre.
	\end{enumerate}
\end{itemize}

Lorsque le mode débogage est activé, le tableau final \emph{n'est pas affiché}. Il est cependant possible de forcer cet affichage en mettant la commande \verbinline-\STdisplaytab- dans l'argument optionnel.

Voici un tableau sur lequel s'appuiera l'exemple suivant :

\begin{lstlisting}
\begin{spreadtab}{{tabular}{|r|r|r|}}\hline
@$x$              &@$y$                & @$x+y$\\\hline\hline
22                & 54                 & \STcopy{v3}{a2+b2} \\
43                & 65                 & \\
49                & 37                 & \\\hline
$Sx=:={a2+a3+a4}$ & $Sy=:={b2+b3+b4}$  & $Sx+Sy=:={}$\\\hline
\multicolumn2{|r|}{$Sy-Sx=:={b5-a5}$}  & @\multicolumn1c{}\\\cline{1-2}
\end{spreadtab}
\end{lstlisting}
\begin{center}
\begin{spreadtab}{{tabular}{|r|r|r|}}\hline
@$x$              &@$y$                & @$x+y$\\\hline\hline
22                & 54                 & \STcopy{v3}{a2+b2} \\
43                & 65                 & \\
49                & 37                 & \\\hline
$Sx=:={a2+a3+a4}$ & $Sy=:={b2+b3+b4}$  & $Sx+Sy=:={}$\\\hline
\multicolumn2{|r|}{$Sy-Sx=:={b5-a5}$}  & @\multicolumn1c{}\\\cline{1-2}
\end{spreadtab}
\end{center}
On va demander à ce que \ST affiche les 3 tableaux de débogage possibles concernant le tableau ci-dessus. Pour cela, il suffit de modifier la ligne 1 du code ci-dessus comme ceci :

\hfill\verbinline[basicstyle=\ttfamily\small]-\begin{spreadtab}[\STdebug{text}\STdebug{formula}\STdebug{code}]{{tabular}{|rr|r|}}\hline-\hfill\kern0pt
\begin{center}
\begin{spreadtab}[\STdebug{text}\STdebug{formula}\STdebug{code}]{{tabular}{|rr|r|}}\hline
@$x$              &@$y$                & @$x+y$\\\hline\hline
22                & 54                 & \STcopy{v3}{a2+b2} \\
43                & 65                 & \\
49                & 37                 & \\\hline
$Sx=:={a2+a3+a4}$ & $Sy=:={b2+b3+b4}$  & $Sx+Sy=:={}$\\\hline
\multicolumn2{|r|}{$Sy-Sx=:={b5-a5}$}  & @\multicolumn1c{}\\\cline{1-2}
\end{spreadtab}
\end{center}
Ces 3 tableaux de débogage peuvent aider à comprendre un peu mieux le fonctionnement interne de \ST. On peut observer dans le tableau 2 que toutes les cellules ayant un champ numérique ont un code interne de 1 ou 2 (voir tableau 3) et ont un marqueur de champ numérique "\verb-:=-" qui leur est associé (voir tableau 1). Ce marqueur représente l'endroit où sera inséré ---~par substitution~--- le résultat du calcul du champ numérique. C'est donc à partir des contenus des champs textuels du tableau 1 et par simple substitution, qu'une fois les champs numériques calculés, les cellules sont reconstituées pour donner celles du tableau final.

Dans les tableaux ci-dessus, les cellules contenant les coordonnées ne sont grisées que si le package \href{http://www.ctan.org/tex-archive/macros/latex/contrib/colortbl/}{\texttt{\textbf{colortbl}}} a été chargé.

\section{Exemples}
Voici quelques tableaux pour finir !

Afin que l'on sache quels nombres sont calculés, seuls les nombres non calculés sont en \textcolor{red}{rouge}. Dans ces tableaux, beaucoup d'artifices (des struts, des multicolumn) et des packages (notamment numprint et ses colonnes \og N\fg{} qui alignent les séparateurs décimaux) ont été utilisés pour obtenir un résultat satisfaisant. Le code est parfois lourd et peu lisible, mais il ne s'agit pas ici d'exemples basiques mais de tableaux peaufinés !

\subsection{Encore un triangle de Pascal}
\begin{lstlisting}
\begin{spreadtab}{{tabular}{*7r}}
[0,1] & [-1,1]+[0,1] & [-1,1]+[0,1] & [-1,1]+[0,1] & [-1,1]+[0,1] & [-1,1]+[0,1] & [-1,1]\\
[0,1] & [-1,1]+[0,1] & [-1,1]+[0,1] & [-1,1]+[0,1] & [-1,1]+[0,1] & [-1,1]       & \\
[0,1] & [-1,1]+[0,1] & [-1,1]+[0,1] & [-1,1]+[0,1] & [-1,1]       &              & \\
[0,1] & [-1,1]+[0,1] & [-1,1]+[0,1] & [-1,1]       &              &              & \\
[0,1] & [-1,1]+[0,1] & [-1,1]       &              &              &              & \\
[0,1] & [-1,1]       &              &              &              &              & \\
\color{red}:={1}&    &              &              &              &              &
\end{spreadtab}
\end{lstlisting}
\begin{center}
\begin{spreadtab}{{tabular}{*7r}}
[0,1] & [-1,1]+[0,1] & [-1,1]+[0,1] & [-1,1]+[0,1] & [-1,1]+[0,1] & [-1,1]+[0,1] & [-1,1]\\
[0,1] & [-1,1]+[0,1] & [-1,1]+[0,1] & [-1,1]+[0,1] & [-1,1]+[0,1] & [-1,1]       & \\
[0,1] & [-1,1]+[0,1] & [-1,1]+[0,1] & [-1,1]+[0,1] & [-1,1]       &              & \\
[0,1] & [-1,1]+[0,1] & [-1,1]+[0,1] & [-1,1]       &              &              & \\
[0,1] & [-1,1]+[0,1] & [-1,1]       &              &              &              & \\
[0,1] & [-1,1]       &              &              &              &              & \\
\color{red}:={1}&    &              &              &              &              &
\end{spreadtab}
\end{center}

\subsection{Convergence d'une série}
Pour les matheux, il s'agit du développement limité de la fonction exponentielle en \numprint{0.5}. En effet,
\[\forall x\in \mathbf{R}\qquad e^x=\sum_{k=0}^\infty\frac{x^k}{k!}\]
et le tableau illustre la rapidité de la convergence au fur et à mesure de l'ordre du développement limité.\par\nobreak
\begin{lstlisting}
\STautoround{15}
\renewcommand\STprintnum[1]{\numprint{#1}}
\begin{spreadtab}{{tabular}{cc}}
\multicolumn{2}{c}{Convergence en $x=\color{red}:={0.5}$}\\[1.5ex]
@$n$            & e^a1\SThidecol                & @ $\displaystyle e^{\numprint{<<a1>>}}-\sum_{k=0}^n\frac{\numprint{<<a1>>}^k}{k!}$\\[3ex]\hline
\color{red}:={0}& a1^[-1,0]/fact([-1,0])        & \STcopy{v}{b!2-b3}\\
\STcopy{v}{a3+1}& \STcopy{v}{a!1^a4/fact(a4)+b3}&                   \\
                &                               &                   \\
                &                               &                   \\
                &                               &                   \\
                &                               &                   \\
                &                               &                   \\
                &                               &                   \\
                &                               &                   \\
                &                               &                   \\\hline
\end{spreadtab}
\end{lstlisting}
\begin{center}
\STautoround{15}
\renewcommand\STprintnum[1]{\numprint{#1}}
\begin{spreadtab}{{tabular}{cc}}
\multicolumn{2}{c}{Convergence en $x=\color{red}:={0.5}$}\\[1.5ex]
@$n$            & e^a1\SThidecol                & @ $\displaystyle e^{\numprint{<<a1>>}}-\sum_{k=0}^n\frac{\numprint{<<a1>>}^k}{k!}$\\[3ex]\hline
\color{red}:={0}& a1^[-1,0]/fact([-1,0])        & \STcopy{v}{b!2-b3}\\
\STcopy{v}{a3+1}& \STcopy{v}{a!1^a4/fact(a4)+b3}&                   \\
                &                               &                   \\
                &                               &                   \\
                &                               &                   \\
                &                               &                   \\
                &                               &                   \\
                &                               &                   \\
                &                               &                   \\
                &                               &                   \\\hline
\end{spreadtab}
\end{center}

\subsection{Convergence vers le nombre d'or}
Voici la définition des nombres de Fibonacci : $F_0=1\qquad F_1=1\qquad F_{n+2}=F_{n+1}+F_n$

On va mettre ici en évidence que le quotient de 2 nombres de Fibonacci consécutifs $F_n$ et $F_{n-1}$ tend vers le nombre d'or $\varphi=\frac{1+\sqrt{5}}{2}$ de telle sorte que la suite $u_n=\varphi-\frac{F_n}{F_{n-1}}$ soit alternativement positive et négative.\par\nobreak
\begin{lstlisting}
\STautoround{9}
$\begin{spreadtab}{{array}{ccN39N{3}{9}}}
@n               & @F_n              & @\hfill{\dfrac{F_n}{F_{n-1}}}\hfill\null& @\hfill{\varphi-\dfrac{F_n}{F_{n-1}}}\hfill\null\\[2ex]\hline
\color{red}:=1   & \color{red}:=1    &                   &                      \\
\STcopy{v}{a2+1} & \color{red}:=1    & \STcopy{v}{b3/b2} & (1+5^0.5)/2-[-1,0]   \\
                 & \STcopy{v}{b2+b3} &                   & \STcopy{v}{d!3+1-c4} \\
                 &                   &                   &                      \\
                 &                   &                   &                      \\
                 &                   &                   &                      \\
                 &                   &                   &                      \\
                 &                   &                   &                      \\
                 &                   &                   &                      \\
                 &                   &                   &                      \\
                 &                   &                   &                      \\
                 &                   &                   &                      \\
                 &                   &                   &                      \\
                 &                   &                   &                      \\
                 &                   &                   &                      \\
                 &                   &                   &                      \\
                 &                   &                   &                      \\\hline
\end{spreadtab}$
\end{lstlisting}
\begin{center}
\STautoround{9}
$\begin{spreadtab}{{array}{ccN39N{3}{9}}}
@n               & @F_n              & @\hfill{\dfrac{F_n}{F_{n-1}}}\hfill\null& @\hfill{\varphi-\dfrac{F_n}{F_{n-1}}}\hfill\null\\[2ex]\hline
\color{red}:=1   & \color{red}:=1    &                   &                      \\
\STcopy{v}{a2+1} & \color{red}:=1    & \STcopy{v}{b3/b2} & (1+5^0.5)/2-[-1,0]   \\
                 & \STcopy{v}{b2+b3} &                   & \STcopy{v}{d!3+1-c4} \\
                 &                   &                   &                      \\
                 &                   &                   &                      \\
                 &                   &                   &                      \\
                 &                   &                   &                      \\
                 &                   &                   &                      \\
                 &                   &                   &                      \\
                 &                   &                   &                      \\
                 &                   &                   &                      \\
                 &                   &                   &                      \\
                 &                   &                   &                      \\
                 &                   &                   &                      \\
                 &                   &                   &                      \\
                 &                   &                   &                      \\
                 &                   &                   &                      \\\hline
\end{spreadtab}$
\end{center}

\subsection{Tableau de facturation}
Voici un tableau de facturation, où les séparateurs décimaux sont alignés dans les colonnes grâce au spécificateur de colonne \og N\fg{} du package \falseverb{numprint}.

Ce tableau est généré par l'environnement \verb=tabularx= de façon à occuper 80\% de la largeur de la ligne. La commande \verbinline=\multicolumn= a été largement utilisée pour la mise en forme :\par\nobreak
\begin{lstlisting}
\nprounddigits2
\let\PC\%
\newcommand\Mystrut{\rule[-1.2ex]{0pt}{4ex}}
\newcommand\RED{\color{red}}
\begin{spreadtab}{{tabularx}{0.8\linewidth}{|>\Mystrut X>\RED N42>\RED c N42>\RED c<\PC N42|}}
\hline
@D\'esignation &@\multicolumn1c{Prix U}& @\multicolumn1c{Qt\'e} & @\multicolumn1c{Prix} & @\multicolumn1c{R\'eduction} & @\textbf{\`A payer}\\\hline
@Item 1 & 5.99 & 20 & [-2,0]*[-1,0] & $-:={20}$ & [-2,0]*(1-[-1,0]/100)\\
@Item 2 & 12   & 7  & [-2,0]*[-1,0] & $-:={10}$ & [-2,0]*(1-[-1,0]/100)\\
@Item 3 & 4.50 & 40 & [-2,0]*[-1,0] & $-:={35}$ & [-2,0]*(1-[-1,0]/100)\\
@Item 4 & 650  & 2  & [-2,0]*[-1,0] & $-:={15}$ & [-2,0]*(1-[-1,0]/100)\\\hline
@\multicolumn6c{}\\[-1.5ex]\cline{4-6}% ligne vide et on remonte un peu !
@\multicolumn1c{\Mystrut}&@\multicolumn2{r|}{\textbf{Total}}& sum(d2:[0,-2]) & \multicolumn1c{$:={round(([1,0]/[-1,0]-1)*100,0)}\PC$} & {\fontseries{b}\selectfont}:={sum(f2:[0,-2])}\\
\cline{4-6}
\end{spreadtab}
\end{lstlisting}
\begin{center}
\nprounddigits2
\let\PC\%
\newcommand\Mystrut{\rule[-1.2ex]{0pt}{4ex}}
\newcommand\RED{\color{red}}
\begin{spreadtab}{{tabularx}{0.8\linewidth}{|>\Mystrut X>\RED N42>\RED c N42>\RED c<\PC N42|}}
\hline
@D\'esignation &@\multicolumn1c{Prix U}& @\multicolumn1c{Qt\'e} & @\multicolumn1c{Prix} & @\multicolumn1c{R\'eduction} & @\textbf{\`A payer}\\\hline
@Item 1 & 5.99 & 20 & [-2,0]*[-1,0] & $-:={20}$ & [-2,0]*(1-[-1,0]/100)\\
@Item 2 & 12   & 7  & [-2,0]*[-1,0] & $-:={10}$ & [-2,0]*(1-[-1,0]/100)\\
@Item 3 & 4.50 & 40 & [-2,0]*[-1,0] & $-:={35}$ & [-2,0]*(1-[-1,0]/100)\\
@Item 4 & 650  & 2  & [-2,0]*[-1,0] & $-:={15}$ & [-2,0]*(1-[-1,0]/100)\\\hline
@\multicolumn6c{}\\[-1.5ex]\cline{4-6}% ligne vide et on remonte un peu !
@\multicolumn1c{\Mystrut}&@\multicolumn2{r|}{\textbf{Total}}& sum(d2:[0,-2]) & \multicolumn1c{$:={round(([1,0]/[-1,0]-1)*100,0)}\PC$} & {\fontseries{b}\selectfont}:={sum(f2:[0,-2])}\\
\cline{4-6}
\end{spreadtab}
\end{center}

\subsection{Carré magique}
\begin{lstlisting}
\begin{spreadtab}{{tabular}{|*3{>{\hfill\rule[-0.4cm]{0pt}{1cm}$}m{0.7cm}<{$\hfill\null}|}}}
\hline
\color{red}:=2 & 5*b2-4*a1         & 3*a1-2*b2 \\\hline
2*a1-b2        & \color{red}:={-1} & 3*b2-2*a1 \\\hline
4*b2-3*a1      & 4*a1-3*b2         & 2*b2-a1   \\\hline
\end{spreadtab}
\end{lstlisting}
\begin{center}
\begin{spreadtab}{{tabular}{|*3{>{\hfill\rule[-0.4cm]{0pt}{1cm}$}m{0.7cm}<{$\hfill\null}|}}}
\hline
\color{red}:=2 & 5*b2-4*a1         & 3*a1-2*b2 \\\hline
2*a1-b2        & \color{red}:={-1} & 3*b2-2*a1 \\\hline
4*b2-3*a1      & 4*a1-3*b2         & 2*b2-a1   \\\hline
\end{spreadtab}
\end{center}

\subsection{Pyramide additive}
Chaque nombre est la somme des deux nombres se trouvant au dessous de lui.\par\nobreak
\begin{lstlisting}
\newlength\cellsize
\setlength\cellsize{1.5cm}
\newcolumntype{K}{@{}>{\rule{0pt}{2.5ex}\centering\arraybackslash$}p{\cellsize}<$@{}}
\begin{spreadtab}{{tabular}{*{8}{@{}p{.5\cellsize}@{}}}}
\cline{4-5}
&&&\multicolumn{2}{|K|}{:={[-1,1]+[1,1]}}&&&\\\cline{3-6}
&&\multicolumn{2}{|K}{:={[-1,1]+[1,1]}}&\multicolumn{2}{|K|}{:={[-1,1]+[1,1]}}&&\\\cline{2-7}
&\multicolumn{2}{|K}{:={[-1,1]+[1,1]}}&\multicolumn{2}{|K}{:={[-1,1]+[1,1]}}&\multicolumn{2}{|K|}{:={[-1,1]+[1,1]}}&\\\hline
\multicolumn{2}{|K}{\color{red}:={-5}}&\multicolumn{2}{|K}{\color{red}:={3}}&\multicolumn{2}{|K}{\color{red}:={-2}}&\multicolumn{2}{|K|}{\color{red}:={-3}}\\\hline
\end{spreadtab}
\end{lstlisting}
\begin{center}
\newlength\cellsize
\setlength\cellsize{1.5cm}
\newcolumntype{K}{@{}>{\rule{0pt}{2.5ex}\centering\arraybackslash$}p{\cellsize}<$@{}}
\begin{spreadtab}{{tabular}{*{8}{@{}p{.5\cellsize}@{}}}}
\cline{4-5}
&&&\multicolumn{2}{|K|}{:={[-1,1]+[1,1]}}&&&\\\cline{3-6}
&&\multicolumn{2}{|K}{:={[-1,1]+[1,1]}}&\multicolumn{2}{|K|}{:={[-1,1]+[1,1]}}&&\\\cline{2-7}
&\multicolumn{2}{|K}{:={[-1,1]+[1,1]}}&\multicolumn{2}{|K}{:={[-1,1]+[1,1]}}&\multicolumn{2}{|K|}{:={[-1,1]+[1,1]}}&\\\hline
\multicolumn{2}{|K}{\color{red}:={-5}}&\multicolumn{2}{|K}{\color{red}:={3}}&\multicolumn{2}{|K}{\color{red}:={-2}}&\multicolumn{2}{|K|}{\color{red}:={-3}}\\\hline
\end{spreadtab}
\end{center}
\medskip
\parskip0pt
\begin{center}
$\star$\par
$\star\quad\star$
\end{center}
C'est tout, j'espère que cette extension vous sera utile !\par\nobreak\medskip
Compte tenu de la jeunesse de ce package, soyez indulgent quant aux dysfonctionnements rencontrés : avec les remarques des utilisateurs, des versions plus abouties verront le jour. En effet, cette version n'est qu'un début et l'ensemble, quoique plutôt stable, reste encore imparfait.\par\nobreak\medskip
Je vous remercie donc d'avance de me signaler par \href{mailto:unbonpetit@gmail.com}{\texttt{\textbf{email}}} tout bug, toute macro-fonction à implémenter que vous pensez utile ou toute proposition d'amélioration \emph{réaliste} : il ne faut pas oublier que cette extension doit rester modeste, que \ST n'est pas excel ou calc et qu'il est impossible d'implémenter toutes les fonctionnalités avancées de ces tableurs.\par\nobreak\bigskip
Christian \textsc{Tellechea}
\end{document}